\documentclass[fleqn]{article}
\oddsidemargin 0.0in
\textwidth 6.0in
\thispagestyle{empty}
\usepackage{import}
\usepackage{amsmath}
\usepackage{graphicx}
\usepackage{flexisym}
\usepackage{amssymb}
\usepackage{bigints} 
\usepackage[english]{babel}
\usepackage[utf8x]{inputenc}
\usepackage{float}
\usepackage[colorinlistoftodos]{todonotes}

\definecolor{hwColor}{HTML}{AD53BA}

\begin{document}

  \begin{titlepage}

    \newcommand{\HRule}{\rule{\linewidth}{0.5mm}}

    \center


    \textsc{\LARGE Arizona State University}\\[1.5cm]

    \textsc{\LARGE Quantum Physics I }\\[1.5cm]


    \begin{figure}
      \includegraphics[width=\linewidth]{asu.png}
    \end{figure}


    \HRule \\[0.4cm]
    { \huge \bfseries Homework Eight}\\[0.4cm] 
    \HRule \\[1.5cm]

    \textbf{Behnam Amiri}

    \bigbreak

    \textbf{Prof: Richard Kirian}

    \bigbreak


    \textbf{{\large \today}\\[2cm]}

    \vfill 

  \end{titlepage}

  \textbf{2.25} \\ \\
  Check that the bound state of the delta-function well (Equation 2.132) is orthogonal to the 
  scattering states (Equations 2.134 and 2.135).

    \textcolor{hwColor}{
      \\\\
      Equation (2.132) 
      $$\psi(x)=\dfrac{\sqrt{m\alpha}}{\hbar} e^{-m\alpha|x|/\hbar^2}$$
      and equations 2.134 and 2.135 are respectively $\psi(x)=A e^{ikx}+B e^{-ikx}$ and $\psi(x)=F e^{ikx}+G e^{-ikx}$. \\
      \\
      Checking the orthogonality: \\
      \\
      $
        \bigints_{-\infty}^{+\infty} \psi^*_m \psi_n dx=\bigints_{-\infty}^{0} \psi^*_m \psi_n dx+\bigints_{0}^{+\infty} \psi^*_m \psi_n dx \\
        \\
        \\
        =\bigints_{-\infty}^{0} \left(\dfrac{\sqrt{m\alpha}}{\hbar} e^{m\alpha x/\hbar^2}\right)^* \left(A e^{ikx}+B e^{-ikx}\right) dx
        +\bigints_{0}^{+\infty} \left(\dfrac{\sqrt{m\alpha}}{\hbar} e^{-m\alpha x/\hbar^2}\right)^* \left(F e^{ikx}+G e^{-ikx}\right) dx \\
        \\
        \\
        =\bigints_{-\infty}^{0} \left(\dfrac{\sqrt{m\alpha}}{\hbar} e^{m\alpha x/\hbar^2}\right) \left(A e^{ikx}+B e^{-ikx}\right) dx
        +\bigints_{0}^{+\infty} \left(\dfrac{\sqrt{m\alpha}}{\hbar} e^{-m\alpha x/\hbar^2}\right) \left(F e^{ikx}+G e^{-ikx}\right) dx \\
        \\
        \\
        =\dfrac{\sqrt{m\alpha}}{\hbar} \left[\bigints_{-\infty}^{0} \left(e^{m\alpha x/\hbar^2}\right) \left(A e^{ikx}+B e^{-ikx}\right) dx
        +\bigints_{0}^{+\infty} \left(e^{-m\alpha x/\hbar^2}\right) \left(F e^{ikx}+G e^{-ikx}\right) dx \right] \\
        \\
        \\
        =\dfrac{\sqrt{m\alpha}}{\hbar} (
          \dfrac{A}{\dfrac{m \alpha}{\hbar^2}+ik}e^{(m\alpha/\hbar^2+ik)x}\Big|_{-\infty}^{0}
          +\dfrac{B}{\dfrac{m \alpha}{\hbar^2}-ik}e^{(m\alpha/\hbar^2-ik)x}\Big|_{-\infty}^{0} \\
          +\dfrac{F}{-\dfrac{m \alpha}{\hbar^2}+ik}e^{(-m\alpha/\hbar^2+ik)x}\Big|_{-\infty}^{0}
          +\dfrac{G}{-\dfrac{m \alpha}{\hbar^2}-ik}e^{(-m\alpha/\hbar^2-ik)x}\Big|_{-\infty}^{0}
        ) \\
        \\
        \\
        =\dfrac{\sqrt{m\alpha}}{\hbar} \left[\dfrac{\dfrac{m \alpha}{\hbar^2}(A+B+F+G)+(-A+B+F-G)ik}{k^2+\left(\dfrac{m \alpha}{h^2}\right)^2}\right] \\
        \\
        \\
      $
      From page 66 of the textbook (Griffiths), the continuity of $\psi(x)$ at $x=0$ requires that: $F+G=A+B$, hence: \\
      \\
      \\
      $
        \Rightarrow \bigints_{-\infty}^{+\infty} \psi^*_m \psi_n dx=\dfrac{\sqrt{m\alpha}}{\hbar} \left[\dfrac{\dfrac{2m \alpha}{\hbar^2}(A+B)-\dfrac{2m \alpha}{\hbar^2}(A+B)}{k^2+\left(\dfrac{m \alpha}{h^2}\right)^2}\right] \\
        \\
        \\
        \\
        \therefore ~~~ \bigints_{-\infty}^{+\infty} \psi^*_m \psi_n dx=0 ~ \surd \\ \\
      $
      Since we got a zero, we can tell that the delta-function is orthogonal to the scattering states. \\
    }

  \rule{15cm}{1pt}

  \textbf{2.27}
  Consider the \emph{double delta-function} potential \\
  $$V(x)=-\alpha \left[\delta(x+a)+ \delta(x-a)\right]$$
  where $\alpha$ and $\beta$ are positive constants. \\ \\
  \begin{itemize}
    \item Sketch this potential.
    

    \item How many bound states does it possess? Find the allowed energies, for 
    $\alpha =\dfrac{\hbar^2}{ma}$ and for $\alpha=\dfrac{\hbar^2}{4ma}$, and sketch 
    the wave functions.

    \item What are the bound state energies in the limiting cases $(i) ~ a \rightarrow  0$
    and $(ii) ~ a \rightarrow \infty$ (holding $\alpha$ fixed?) Explain why your answers are reasonable, by comparison with the single
    delta-function well. 
  \end{itemize}

    \textcolor{hwColor}{
      \\
      Insert a picture here \\
      \\
      \\
      \\
      Let's start again with the Schr$\ddot{o}$dinger equation. \\
      \\
      \\
      $i\hbar \dfrac{\partial \Psi}{\partial t}=-\dfrac{\hbar^2}{2m}\dfrac{\partial^2 \Psi}{\partial x}+V(x,t) \Psi(x,t), ~~~~~ -\infty<x<+\infty, ~~ t>0$ \\
      \\
      \\
      We have $V(x)=-\alpha \left[\delta(x+a)+ \delta(x-a)\right]$, let's plug it into the above equation: \\
      \\
      \\
      $
        i\hbar \dfrac{\partial \Psi}{\partial t}=-\dfrac{\hbar^2}{2m}\dfrac{\partial^2 \Psi}{\partial x}-\alpha \left[\delta(x+a)+ \delta(x-a)\right] \Psi(x,t)
      $ \\
      \\
      By the method of separation of variables: \\
      \\
      \\
      $
        \Psi(x,t)=\psi(x) \phi(t) \Longrightarrow i\hbar \dfrac{\partial}{\partial t} \left[\psi(x) \phi(t)\right]=-\dfrac{\hbar^2}{2m}\dfrac{\partial^2}{\partial x}\left[\psi(x) \phi(t)\right]-\alpha \left[\delta(x+a)+ \delta(x-a)\right] \left[\psi(x) \phi(t)\right] \\ \\
        \\
        \\
        i \hbar \psi(x) \phi^'(t)=-\dfrac{\hbar^2}{2m} \psi^{''}(x) \phi(t)-\alpha \psi(x) \phi(t) \left[\delta(x+a)+\delta(x-a)\right] \\
        \\
        \\
        i \hbar \dfrac{\phi^'(t)}{\phi(t)}=-\dfrac{\hbar^2}{2m} \dfrac{\psi^{''}(x)}{\psi(x)}- \alpha \left[\delta(x+a)+\delta(x-a)\right] \\
        \\
        \\
      $
      On the left side we have a function of t and on the other side we have a function of x. Therefore, the only way are equal if they are both 
      equal to a constant. Let's call that constant $K$. \\
      \\
      $
        i \hbar \dfrac{\phi^'(t)}{\phi(t)}=-\dfrac{\hbar^2}{2m} \dfrac{\psi^{''}(x)}{\psi(x)}- \alpha \left[\delta(x+a)+\delta(x-a)\right]=K \\
        \\
        \\
        \begin{cases}
          K=i \hbar \dfrac{\phi^'(t)}{\phi(t)} \\
          \\
          K=-\dfrac{\hbar^2}{2m} \dfrac{\psi^{''}(x)}{\psi(x)}- \alpha \left[\delta(x+a)+\delta(x-a)\right]
        \end{cases} \\ 
      $
      \\
      Mathematically speaking, $K$ for which the boundary conditions are satisfied are called the eigenvalues
      and the nontrivial solutions associated with them are called the eigenfunctions. 
      The ODE in x is known as the time-independent Schr$\ddot{o}$dinger equation. \\
      \\
      \\
      $
        \dfrac{d^2 \psi}{dx^2}+ \psi(x)\dfrac{2m}{\hbar^2} \left[\alpha \left(\delta(x+a)+\delta(x-a)\right)+K\right]=0 ~~~~~~~~~~~~~~~ (A) \\
        \\
        \\
        \dfrac{d^2 \psi}{dx^2}+\dfrac{2m \psi K}{\hbar^2}=0, ~~~ x\neq a, -a \\
        \\
        \\
        \psi(x)=\begin{cases}
          C_1 exp\left(\dfrac{\sqrt{-2mK}x}{\hbar}\right)+C_2 exp\left(-\dfrac{\sqrt{-2mK}x}{\hbar}\right), ~~~~~ x<-a \\
          \\
          C_3 exp\left(\dfrac{\sqrt{-2mK}x}{\hbar}\right)+C_4 exp\left(-\dfrac{\sqrt{-2mK}x}{\hbar}\right), ~~~~~ -a<x<+a\\
          \\
          C_5 exp\left(\dfrac{\sqrt{-2mK}x}{\hbar}\right)+C_6 exp\left(-\dfrac{\sqrt{-2mK}x}{\hbar}\right), ~~~~~ x>a
        \end{cases}
        \\
        \\
        \\
      $
      To make our lives easier, let's set $C_2$ and $C_5$ equal to zero. Therefore, \\
      \\
      \\
      $
        \psi(x)=\begin{cases}
          C_1 exp\left(\dfrac{\sqrt{-2mK}x}{\hbar}\right), ~~~~~ x<-a \\
          \\
          C_3 exp\left(\dfrac{\sqrt{-2mK}x}{\hbar}\right)+C_4 exp\left(-\dfrac{\sqrt{-2mK}x}{\hbar}\right), ~~~~~ -a<x<+a\\
          \\
          C_6 exp\left(-\dfrac{\sqrt{-2mK}x}{\hbar}\right), ~~~~~ x>a
        \end{cases} \\
        \\
        \\
        \\
        \begin{cases}
          \lim\limits_{x\to -a^+} \psi(x)=\lim\limits_{x\to -a^-} \psi(x): ~~~ C_1 exp\left(-\dfrac{\sqrt{-2mK}a}{\hbar}\right)=C_3 exp\left(-\dfrac{\sqrt{-2mK}a}{\hbar}\right)+C_4 exp\left(\dfrac{\sqrt{-2mK}a}{\hbar}\right) \\
          \\
          \lim\limits_{x\to a^+} \psi(x)=\lim\limits_{x\to a^-} \psi(x): ~~~ C_6 exp\left(-\dfrac{\sqrt{-2mK}a}{\hbar}\right)=C_3 exp\left(\dfrac{\sqrt{-2mK}a}{\hbar}\right)+C_4 exp\left(-\dfrac{\sqrt{-2mK}a}{\hbar}\right)
        \end{cases} 
        \\
        \\
        \\
      $
      Taking the integral of (A), \\ \\ 
      $
        \bigints_{-a-\epsilon}^{-a+\epsilon} \left[\dfrac{d^2 \psi}{dx^2}+\dfrac{2m}{\hbar^2} \psi(x) \left(\alpha\left(\delta(x+a)+\delta(x-a)+K\right)\right)\right]dx=0 \\
        \\
        \\
        \bigints_{-a-\epsilon}^{-a+\epsilon} \dfrac{d^2 \psi}{dx^2} dx+\dfrac{2m}{\hbar^2} \left[
          \alpha \bigints_{-a-\epsilon}^{-a+\epsilon} \psi(x) \left[\delta(x+a)+\delta(x-a)\right] dx
          +\bigints_{-a-\epsilon}^{-a+\epsilon} \psi(x) K dx
        \right]=0 \\
        \\
        \\
        \dfrac{d\psi}{dx} \Big|_{-a-\epsilon}^{-a+\epsilon}+\dfrac{2m}{\hbar^2} \left[-\alpha \psi a-2K\psi \epsilon \right]=0 \\
        \\
        \\
      $
      Taking the limit; \\
      \\
      $
        \dfrac{d\psi}{dx} \Big|_{-a-\epsilon}^{-a+\epsilon}+\dfrac{2m \alpha \psi(-a)}{\hbar^2}=0 \Rightarrow \dfrac{2m \alpha \psi(-a)}{\hbar^2}=\lim\limits_{x\to -a^-} \dfrac{\psi(x)}{dx}-\lim\limits_{x\to -a^+} \dfrac{\psi(x)}{dx} \\
        \\
        \\
        \\
        C_1 \dfrac{2m \alpha}{\hbar} exp\left(-\dfrac{\sqrt{-2mK}a}{\hbar}\right)=
        C_1 \dfrac{\sqrt{-2mK}a}{\hbar} \left(-\dfrac{\sqrt{-2mK}a}{\hbar}\right) \\ \\
        -\left[
          C_3 \dfrac{\sqrt{-2mK}a}{\hbar} exp\left(-\dfrac{\sqrt{-2mK}a}{\hbar}\right)
          -C_4 \dfrac{\sqrt{-2mK}a}{\hbar} exp\left(\dfrac{\sqrt{-2mK}a}{\hbar}\right)
        \right] \\ \\
      $
      \rule{15cm}{1pt}
      \\
      \\
      The final condition: \\
      \\
      \\
      $
        \bigints_{a-\epsilon}^{a+\epsilon} \left[\dfrac{d^2 \psi}{dx^2}+\dfrac{2m}{\hbar^2} \psi(x) \left(\alpha\left(\delta(x+a)+\delta(x-a)+K\right)\right)\right]dx=0 \\
        \\
        \\
        \bigints_{a-\epsilon}^{a+\epsilon} \dfrac{d^2 \psi}{dx^2} dx+\dfrac{2m}{\hbar^2} \left[
          \alpha \bigints_{a-\epsilon}^{a+\epsilon} \psi(x) \left[\delta(x+a)+\delta(x-a)\right] dx
          +\bigints_{a-\epsilon}^{a+\epsilon} \psi(x) K dx
        \right]=0 \\
        \\
        \\
        \\
        \dfrac{d\psi}{dx} \Big|_{a-\epsilon}^{a+\epsilon}+\dfrac{2m}{\hbar^2} \left[\alpha \psi(a)+2K \psi(a) \epsilon\right]=0
        \\
        \\
      $
      Taking the limit; \\
      \\
      \\
      $
        \dfrac{d\psi}{dx} \Big|_{a-\epsilon}^{a+\epsilon}+\dfrac{2m \alpha \psi(a)}{\hbar^2}=0 \Rightarrow \dfrac{2m \alpha \psi(a)}{\hbar^2}=\lim\limits_{x\to a^-} \dfrac{\psi(x)}{dx}-\lim\limits_{x\to a^+} \dfrac{\psi(x)}{dx} \\
        \\
        \\
        \\
        C_6 \dfrac{2m \alpha}{\hbar^2} exp\left(-\dfrac{\sqrt{-2mK}a}{\hbar}\right)= 
        +C_6 \dfrac{\sqrt{-2mK}a}{\hbar} exp\left(-\dfrac{\sqrt{-2mK}a}{\hbar}\right) \\ \\
        +\left[
          C_3 \dfrac{\sqrt{-2mK}a}{\hbar} exp \left(\dfrac{\sqrt{-2mK}a}{\hbar}\right)
          -C_4 \dfrac{\sqrt{-2mK}a}{\hbar} exp\left(-\dfrac{\sqrt{-2mK}a}{\hbar}\right)
        \right]
      $
      \\
      \\
      \\
      Phew, now we have four equations. Let's list them all here: \\
      \\
      \\
      $
        \begin{cases}
          C_1 e^{-w a}-C_3 e^{-w a}-C_4 e^{wa}=0 \\
          \\
          C_3 e^{w a}+C_4 e^{-w a}-C_6 e^{-wa}=0 \\
          \\
          C_1 \dfrac{2m \alpha}{\hbar^2} e^{-w a}-C_1 w e^{-w a}+C_3 we^{-w a}-C_4 w e^{w a}=0 \\
          \\
          C_6 \dfrac{2m \alpha}{\hbar^2} e^{-w a}-C_3 w e^{w a}+C_4 w e^{-w a}-C_6 w e^{-w a}=0 
        \end{cases}, ~~~ where ~~~~ k=\dfrac{\sqrt{-2mK}}{\hbar}
      $
      \\
      \\
      $
        \begin{cases}
          C_1 e^{-w a} \left(\dfrac{2m \alpha}{\hbar^2}-w\right)=-C_3 we^{-w a}+C_4 w e^{w a} \\
          \\
          C_6 e^{-w a} \left(\dfrac{2m \alpha}{\hbar^2}-w\right)=C_3 w e^{w a}-C_4 w e^{-w a} 
        \end{cases} \\
        \\
        \\
        \begin{cases}
           \left(C_3 e^{-w a}+C_4 e^{w a}\right) \left(\dfrac{2m \alpha}{\hbar^2}-w\right)=-C_3 we^{-w a}+C_4 w e^{w a} \\
           \\
           \left(C_3 e^{w a}+C_4 e^{-w a}\right) \left(\dfrac{2m \alpha}{\hbar^2}-w\right)=C_3 w e^{w a}-C_4 w e^{-w a} 
        \end{cases} \Longrightarrow 
        \begin{cases}
          C_4 \dfrac{k \hbar^2}{m \alpha}=C_3 e^{-2 w a}+C_4 \\
          \\
          C_3 \dfrac{k \hbar^2}{m \alpha}=C_3 +C_4 e^{-2w a}
        \end{cases} \\
        \\
        \\
        \begin{cases}
          C_3=C_4 \left(\dfrac{w \hbar^2}{m \alpha}-1\right) e^{2 w a} \\
          \\
          C_4=C_3 \left(\dfrac{w \hbar^2}{m \alpha}-1\right) e^{2 w a}
        \end{cases}
      $ \\
      \\
      \\
      \\
      From the above results we have: $C_4=C_4 e^{4wa} \left(\dfrac{w \hbar^2}{m \alpha}-1\right)$. With this equation we can find the values of K (the allowed energies).
      We are told that $\alpha=\dfrac{\hbar^2}{ma}$, therefore: \\
      \\
      \\
      $
        \left(wa-1\right)^2 e^{4wa} \Longrightarrow \begin{cases}
          \dfrac{\sqrt{-2mK}a}{\hbar} \approx 0.79 \Rightarrow K=-\dfrac{0.31 \hbar^2}{ma^2} \\
          \\
          \dfrac{\sqrt{-2mK}a}{\hbar}=0 \Rightarrow K=0 \\
          \\
          \dfrac{\sqrt{-2mK}a}{\hbar} \approx 1.10 \Rightarrow K=-\dfrac{0.61 \hbar^2}{ma^2} \\
        \end{cases}
      $ \\
      \\
      From the textbook (Griffiths) we learned that when an energy is negative, then there are two bound states for the value of $\alpha$, consequently: \\
      \\
      \\
      $
        \psi(x)=\begin{cases}
          C_1 exp\left(\dfrac{\sqrt{-2mK}x}{\hbar}\right), ~~~~~ x<-a \\
          \\
          C_3 exp\left(\dfrac{\sqrt{-2mK}x}{\hbar}\right)+C_4 exp\left(-\dfrac{\sqrt{-2mK}x}{\hbar}\right), ~~~~~ -a<x<+a \\
          \\
          C_6 exp\left(-\dfrac{\sqrt{-2mK}x}{\hbar}\right), ~~~~~ x>a
        \end{cases} \\
        \\
        \\
        =\begin{cases}
          \left(C_3+C_4 e^{2wa}\right) exp\left(\dfrac{0.79x}{a}\right), ~~~~~ x<-a \\
          \\
          C_3 exp\left(\dfrac{0.79x}{a}\right)+C_4 exp\left(-\dfrac{0.79x}{a}\right), ~~~~~ -a<x<+a \\
          \\
          \left(C_3 e^{2wa}+C_4\right) exp\left(-\dfrac{0.79x}{a}\right), ~~~~~ x>a
        \end{cases} \\
        \\
        \\
        =\begin{cases}
          \left[C_4 e^{2wa}\left(\dfrac{\hbar^2 w}{ma}-1\right)+C_4 e^{2wa}\right] exp\left(\dfrac{0.79 x}{a}\right), ~~~~~ x<-a \\
          \\
          C_4 e^{2wa}\left(\dfrac{\hbar^2 w}{ma}-1\right) exp\left(\dfrac{0.79 x}{a}\right)+C_4 exp\left(-\dfrac{0.79x}{a}\right),  ~~~~~ -a<x<+a \\
          \\
          \left[C_4 e^{4wa} \left(\dfrac{\hbar^2 w}{ma}-1\right)+C_4\right]exp\left(-\dfrac{0.79x}{a}\right), ~~~~~ x>a
        \end{cases} \\
        \\
        \\
        \\
        \\
        \therefore ~~~ \psi(x)=\begin{cases}
          3.9 C_4 exp\left(\dfrac{0.79x}{a}\right), ~~~~~ x<-a \\
          \\
          C_4 \left[exp\left(-\dfrac{0.79x}{a}\right)-exp\left(\dfrac{0.79x}{a}\right)\right], ~~~~~ -a<x<+a \\
          \\
          -3.9 C_4 exp\left(-\dfrac{0.79x}{a}\right), ~~~~~ x>a
        \end{cases} 
      $
      \\
      \\
      \includegraphics[height=6cm, width=12cm]{1.JPG} \\
      \\
      \\
      $C_4$ is a normalization constant. \\
      \\
      \\
      $
        \bigints_{-\infty}^{+\infty} \left[\psi(x)\right]^2 dx=1 \\
        \\
        \\
        =\bigints_{-\infty}^{-a} \left[\psi(x)\right]^2 dx+\bigints_{-a}^{+a} \left[\psi(x)\right]^2 dx+\bigints_{+a}^{+\infty} \left[\psi(x)\right]^2 dx \\
        \\
        \\
        \therefore ~~~ C_4\approx \dfrac{0.4}{\sqrt{a}}
      $
      Therefore, the bound state with $K=-\dfrac{0.31 \hbar^2}{ma^2}$ when $\alpha=\dfrac{\hbar^2}{ma}$ is: \\
      \\
      \\
      $
        \therefore ~~~ \psi(x)=\begin{cases}
          \dfrac{1.62}{\sqrt{a}} exp\left(\dfrac{0.79x}{a}\right), ~~~~~ x<-a \\
          \\
          \dfrac{0.41}{\sqrt{a}} \left[exp\left(-\dfrac{0.79x}{a}\right)-exp\left(\dfrac{0.79x}{a}\right)\right], ~~~~~ -a<x<+a \\
          \\
          -\dfrac{1.62}{\sqrt{a}} exp\left(-\dfrac{0.79x}{a}\right), ~~~~~ x>a
        \end{cases} ~~~~ \surd
      $ \\
      \\
      \\
      \\
      The probability distribution for the particle’s position at time t in this state is: \\
      \\
      $
        |\psi(x) e^{-iKt/\hbar}|^2=\left[\psi(x) e^{-iKt/\hbar}\right] \left[\psi(x) e^{-iKt/\hbar}\right]^*=\left[\psi(x) e^{-iKt/\hbar}\right] \left[\psi(x) e^{iKt/\hbar}\right] \\
        \\
        \\
        \\
        \therefore ~~~ |\psi(x) e^{-iKt/\hbar}|^2=|\psi(x)|^2 ~~ \surd
      $ \\
      \\
      \\
      Below you can see $\psi$  and $|\psi|^2$. \\
      \\
      \\
      \includegraphics[height=6cm, width=12cm]{2.JPG} \\
      \\
      \\
      \includegraphics[height=6cm, width=12cm]{3.JPG}
      \\
      \\
      \rule{15cm}{1pt}
      \\
      \\
      Now, it is time to find the eigenstate for the case where $K \approx -0.61 \dfrac{\hbar^2}{ma^2}$. \\
      \\
      \\
      $
        \psi(x)=\begin{cases}
          C_1 exp\left(\dfrac{\sqrt{-2mK}x}{\hbar}\right), ~~~~~ x<-a \\
          \\
          C_3 exp\left(\dfrac{\sqrt{-2mK}x}{\hbar}\right)+C_4 exp\left(-\dfrac{\sqrt{-2mK}x}{\hbar}\right), ~~~~~ -a<x<+a \\
          \\
          C_6 exp\left(-\dfrac{\sqrt{-2mK}x}{\hbar}\right), ~~~~~ x>a
        \end{cases} \\
        \\
        \\
        \\
        =\begin{cases}
          \left(C_3+C_4e^{2wa}\right) exp \left(\dfrac{1.10x}{a}\right), ~~~~~ x<-a \\
          \\
          C_3 exp\left(\dfrac{1.10x}{a}\right)+C_4 exp\left(-\dfrac{1.10x}{a}\right), ~~~~~ -a<x<+a \\
          \\
          \left(C_3 e^{2wa}+C_4\right) exp \left(-\dfrac{1.10x}{a}\right), ~~~~~ x>a
        \end{cases} \\
        \\
        \\
        \\
        =\begin{cases}
          \left[ C_4e^{2wa}\left(\dfrac{\hbar^2 w}{m \alpha}-1\right)+C_4e^{2wa}\right] exp\left(\dfrac{1.10x}{a}\right), ~~~~~ x<-a \\
          \\
          C_4 e^{2wa}\left(\dfrac{\hbar^2 w}{m \alpha}-1\right) exp\left(\dfrac{1.10x}{a}\right)+C_4 exp\left(-\dfrac{1.10x}{a}\right), ~~~~~ -a<x<+a \\
          \\
          \left[C_4 e^{4wa}\left(\dfrac{\hbar^2 w}{m \alpha}-1\right)+C_4\right]exp\left(-\dfrac{1.10x}{a}\right), ~~~~~ x>a
        \end{cases} \\
        \\
        \\
        \\
        =\begin{cases}
          10.18 C_4 exp\left(\dfrac{1.10x}{a}\right), ~~~~~ x<-a \\
          \\
          C_4 \left[exp\left(\dfrac{1.10x}{a}\right)+exp\left(-\dfrac{1.10x}{a}\right)\right], ~~~~~ -a<x<+a \\
          \\
          10.18 C_4 exp\left(-\dfrac{1.10x}{a}\right), ~~~~~ x>a
        \end{cases} \\
        \\
        \\
        \\
      $
      $C_4$ is a normalization constant. \\
      \\
      \\
      $
        \bigints_{-\infty}^{+\infty} \left[\psi(x)\right]^2 dx=1 \\
        \\
        \\
        =\bigints_{-\infty}^{-a} \left[\psi(x)\right]^2 dx+\bigints_{-a}^{+a} \left[\psi(x)\right]^2 dx+\bigints_{+a}^{+\infty} \left[\psi(x)\right]^2 dx \\
        \\
        \\
        \therefore ~~~ C_4\approx \dfrac{0.2}{\sqrt{a}} \\
        \\
        \\
      $
      Therefore, the bound state with $K=-\dfrac{0.61 \hbar^2}{ma^2}$ when $\alpha=\dfrac{\hbar^2}{ma}$ is: \\
      \\
      \\
      $
        \therefore ~~~ \psi(x)=\begin{cases}
          \dfrac{2.15}{\sqrt{a}} exp\left(\dfrac{1.10x}{a}\right), ~~~~~ x<-a \\
          \\
          \dfrac{0.21}{\sqrt{a}} \left[exp\left(-\dfrac{1.10x}{a}\right)-exp\left(\dfrac{1.10x}{a}\right)\right], ~~~~~ -a<x<+a \\
          \\
          \dfrac{2.15}{\sqrt{a}} exp\left(-\dfrac{1.10x}{a}\right)), ~~~~~ x>a
        \end{cases} ~~~~ \surd
      $
      \\
      \\
      \\
      \\
      \includegraphics[height=6cm, width=12cm]{4.JPG} \\
      \\
      \\
      \includegraphics[height=6cm, width=12cm]{5.JPG}
      \\
      \\
      \rule{15cm}{1pt}
    }

  \rule{15cm}{1pt}

  \textbf{2.34}
  Consider the "step" potential: 
  $$V(x)=\begin{cases}
    0, ~~~ x\leqslant 0 \\
    V_0, ~~~ x > 0
  \end{cases}$$
  \begin{itemize}
    \item Calculate the reflection coefficient, for the case $E>V_0$, and comment on the answer.

    \item Calculate the reflection coefficient for the case $E>V_0$.

    \item For a potential (such as this one) that does not go back to zero to the right of the barrier,
    the transmission coefficient is not simply $|F|^2/|A|^2$ (with A the incident amplitude and F
    the transmitted amplitude), because the transmitted wave travels at a different speed. Show
    that
    $$T=\sqrt{\dfrac{E-V_0}{E}}\dfrac{|F|^2}{|A|^2}$$
    for $E>V_0$. \emph{Hint:} You can figure it out using Equation 2.99, or—more elegantly, but less
    informatively—from the probability current (Problem 2.18). What is $T$, $E<V_0?$

    \item For $E>V_0$, calculate the transmission coefficient for the step potential, and check that $T+R=1$.
    
  \end{itemize}

    \textcolor{hwColor}{
      Let's start with our to-go equation, the Schr$\ddot{o}$dinger equation. \\
      \\
      \\
      $i\hbar \dfrac{\partial \Psi}{\partial t}=-\dfrac{\hbar^2}{2m}\dfrac{\partial^2 \Psi}{\partial x}+V \Psi$ \\
      \\
      For this problem, It's a good choice to use the method of separation of variables. \\
      \\
      \rule{15cm}{1pt} \\
      \\
      \textbf{Quick Review:} \\
      \\
      \textcolor{hwColor}{
        The method of separation of variables relies upon the assumption that a function of the form $u(x,t)=\phi(x) G(t)$. 
        will be a solution to a linear homogeneous partial differential equation in $x$ and $t$. This is called a product solution
        and provided the boundary conditions are also linear and homogeneous this will also satisfy the boundary conditions.
        The method of Separation of Variables cannot always be used and even when it can be used 
        it will not always be possible to get much past the first step in the method. \\
      }
      \\
      \rule{15cm}{1pt}
      \\
      \\
      Alrighty, we assume $\Psi(x,t)=\psi(x) \phi(t)$ is a solution, then we have: \\
      \\
      $
        i\hbar \dfrac{\partial}{\partial t} \left[\psi(x) \phi(t)\right]=-\dfrac{\hbar^2}{2m}\dfrac{\partial^2}{\partial x} \left[\psi(x) \phi(t)\right]+V \left[\psi(x) \phi(t)\right] \\
        \\
        \\
        i\hbar \psi(x) \phi^'(t)=-\dfrac{\hbar^2}{2m} \psi^{''}(x) \phi(t)+V(x) \psi(x) \phi(t) \\
        \\
        \\
        i\hbar \dfrac{\phi^'(t)}{\phi(t)}=-\dfrac{\hbar^2}{2m} \dfrac{\psi^{''}(x)}{\psi(x)}+V(x)
      $
      \\
      \\
      We use $\lambda$ as the separation constant. Therefore, \\
      \\
      $
      i\hbar \dfrac{\phi^'(t)}{\phi(t)}=-\dfrac{\hbar^2}{2m} \dfrac{\psi^{''}(x)}{\psi(x)}+V(x)=\lambda \Rightarrow \begin{cases}
        \lambda=-\dfrac{\hbar^2}{2m} \dfrac{\psi^{''}(x)}{\psi(x)}+V(x)
        \\
        \\
        \lambda=i\hbar \dfrac{\phi^'(t)}{\phi(t)}
      \end{cases} \\ \\
      $
      \\
      \textbf{The eigenvalues:} \\
      \\
      \\
      $
        \dfrac{d^2 \psi}{dx^2}=\dfrac{2m}{\hbar^2}\psi \left[V(x-\lambda)\right] \Rightarrow \begin{cases}
          \dfrac{d^2 \psi}{dx^2}=\dfrac{2m}{\hbar^2} \psi\left(-V_0-\lambda\right), ~~~~ x>0 \\
          \\
          \dfrac{d^2 \psi}{dx^2}=-\dfrac{2m \lambda \psi}{\hbar^2}, ~~~~ x\leq 0
        \end{cases}
      $
      \\
      \\
      \\
      For $\psi$ we have three cases now (when $x>0$):
      $$\begin{cases}
        V_0-\lambda<0
        \\
        V_0-\lambda=0 
        \\
        V_0-\lambda>0
      \end{cases}$$
      \\
      \\
      Time to study each case: \\
      \\
      \textbf{(A) $V_0-\lambda>0$} \\
      \\
      $
        \begin{cases}
          \dfrac{d^2 \psi}{dx^2}=-\dfrac{2m\lambda \psi}{\hbar^2}, ~~~~ x \leq 0 \\
          \\
          \dfrac{d^2 \psi}{dx^2}=\dfrac{2m \psi}{\hbar^2} \left(V_0-\lambda\right), ~~~ x>0
        \end{cases}
      $ And the general solution is 
      $
        \psi(x)=\begin{cases}
          Ae^{ikx}+Be^{-ikx}, ~~~~ x\leq 0
          \\
          Fe^{dx}+Ge^{-dx}, ~~~~ x>0
        \end{cases}
      $ where $k=\dfrac{\sqrt{2m\lambda}}{\hbar}, ~~~~ d=\dfrac{\sqrt{2m(V_0-\lambda)}}{\hbar}$ \\
      \\
      As $x \to \infty$ we have: \\
      \\
      $
        \psi(x)=\begin{cases}
          Ae^{ikx}+Be^{-ikx}, ~~~~ x\leq 0
          \\
          Ge^{-dx}, ~~~~~~~~~~~~~~~~ x>0
        \end{cases}
      $ \\
      \\
      $
        T=\dfrac{\dfrac{i\hbar}{2m} \left[(Ge^{-dx})\dfrac{d}{dx}(G^*e^{-dx})- (G^*e^{-dx})\dfrac{d}{dx}(Ge^{-dx})\right]}{\dfrac{i\hbar}{2m} \left[(Ae^{ikx})\dfrac{d}{dx}(A^*e^{-ikx})-(A^*e^{-ikx})\dfrac{d}{dx}(Ae^{ikx})\right]} \\
        \\
        \\
        =\dfrac{-de^{-2dx}GG^*+de^{-2dx}GG^*}{-2ikAA^*} \\
        \\
        \\
        \therefore ~~~ T=0 ~~ \surd
        \\
        \\
        R=\dfrac{\dfrac{i\hbar}{2m}  \left[(Be^{-ikx})\dfrac{d}{dx}(B^*e^{ikx})-(B^*e^{ikx})\dfrac{d}{dx}(Be^{-ikx})\right]}{\dfrac{i\hbar}{2m} \left[(Ae^{ikx})\dfrac{d}{dx}(A^*e^{-ikx})-(A^*e^{-ikx})\dfrac{d}{dx}(Ae^{ikx})\right]} \\
        \\
        \\
        =\dfrac{2ikBB^*}{-2ikAA^*} \\
        \\
        \\
        \therefore ~~~ R=\dfrac{|B|^2}{|A|^2} \\
        \\
        \\
      $
      Constants determination: \\
      \\
      $
        \lim\limits_{x\to 0^+} \psi(x)=\lim\limits_{x\to 0^-} \psi(x): ~~ A+B=G \\
        \\
        \\
        \bigints_{-\epsilon}^{+\epsilon} \dfrac{d^2 \psi}{dx^2}dx=\bigints_{-\epsilon}^{+\epsilon}  \dfrac{2m}{\hbar^2} \psi(x) \left[V(x)-\lambda\right]dx \\
        \\
        \dfrac{d \psi}{dx} \Big|_{-\epsilon}^{+\epsilon}=-\bigints_{-\epsilon}^{0} \dfrac{2m \lambda \psi(x)}{\hbar^2}dx+\bigints_{0}^{+\epsilon} \dfrac{2m \psi(x)(V_0-\lambda)}{\hbar^2} dx \\
        \\
        =-\dfrac{2m\lambda \psi(0) \epsilon}{\hbar^2}+\dfrac{2m\psi(0)\epsilon (V_0-\lambda)}{\hbar^2} \\
        \\
        \\
        \dfrac{d \psi}{dx} \Big|_{0^-}^{0^+}, \longleftrightarrow \epsilon \rightarrow 0 \\
        \\
        \\
        \lim\limits_{x\to 0^+} \dfrac{d\psi}{dx}=\lim\limits_{x\to 0^-} \dfrac{d\psi}{dx}: ~~ (A-b)ik=-dG \Longrightarrow
        \begin{cases}
          ik(A-b)=-d(A+B) \\
          \\
          B=\dfrac{-d-ik}{d-ik}A
        \end{cases} \\
        \\
        \\
      $
      The reflection coefficient is: \\
      \\
      $
        R=\left(\dfrac{A}{B}\right) \left(\dfrac{B}{A}\right)^*=\left(\dfrac{-d-ik}{d-ik}\right) \left(\dfrac{-d+ik}{d+ik}\right)=\dfrac{d^2+k^2}{d^2+k^2} \\
        \\
        \\
        \therefore ~~~~ R+T=1
      $
      \\
      \\
      \rule{15cm}{1pt}
      \\
      \\
      \textbf{(B) $V_0-\lambda=0$} \\
      \\
      $
        \begin{cases}
          \dfrac{d^2 \psi}{dx^2}=-\dfrac{2mV_0 \psi}{\hbar^2}, ~~~ x \leq 0 \\
          \\
          \dfrac{d^2 \psi}{dx^2}=0, ~~~~~~~~~~~~~~ x>0
        \end{cases} \\ 
      $
      The general solution on $x>0$ is: \\
      \\
      $
        \psi(x)=\begin{cases}
          A e^{ikx}+B e^{-eikx}, ~~~ x \leq 0 \\
          \\
          Fx+G, ~~~~~~~~~~~~~~~ x>0
        \end{cases} where ~ k=\dfrac{\sqrt{2mV_0}}{\hbar} \\ \\
      $ 
      Satisfy the boundary condition as $x \rightarrow \infty$, then set $F$ and $G$ to zero. \\ 
      \\
      $
        \psi(x)=\begin{cases}
          A e^{ikx}+B e^{-eikx}, ~~~ x \leq 0 \\
          \\
          0, ~~~~~~~~~~~~~~~~~~~~~~~~~ x>0
        \end{cases} \\
        \\
        \\
        \begin{cases}
          \lim\limits_{x\to 0^+} \psi(x)=\lim\limits_{x\to 0^-} \psi(x): ~~ A+B=0 \\
          \\
          \lim\limits_{x\to 0^+} \dfrac{d \psi}{dx}=\lim\limits_{x\to 0^-} \dfrac{d\psi}{dx}: ~~ (A-B)ik=G
        \end{cases} \Longrightarrow A=0, B=0, \psi(x)=0 \\ \\
      $ \\ 
      Hence, $E=V_0$ is \textbf{NOT} an eigenvalue! \\
      \\
      \rule{15cm}{1pt}
      \\
      \\
      \textbf{(C) $V_0-\lambda<0$} \\
      \\
      $
        \begin{cases}
          \dfrac{d^2 \psi}{dx^2}=-\dfrac{2m\lambda \psi}{\hbar^2}, ~~~~~~~~~~~~ x \leq 0 \\
          \\
          \dfrac{d^2 \psi}{dx^2}=-\dfrac{2m\psi(\lambda-V_0)}{\hbar^2}, ~~~  x>0
        \end{cases} \\ \\
      $
      The general solution on $x>0$ is: \\
      \\
      $
        \psi(x)=\begin{cases}
          Ae^{ikx}+Be^{-ikx}, ~~~~ x \leq 0 \\
          \\
          Fe^{ilx}+Ge^{-lx}, ~~~~~~ x>0
        \end{cases}
      $ where $l=\dfrac{\sqrt{2m(\lambda-V_0)}}{\hbar}$ and $k=\dfrac{\sqrt{2m\lambda}}{\hbar}$ \\ 
      \\
      Solving the ODE in $t$ yields $\phi(t)=e^{-i\lambda t/\hbar}$, which means the product solution is a linear combination
      of waves travelling to the left and to the right. \\
      \\
      $
        \psi(x) \phi(t)=\begin{cases}
          Ae^{i(kx-\lambda t/\hbar)}+Be^{-i(kx+\lambda t/\hbar)}, ~~~ x \leq 0 \\
          \\
          F e^{i(\ell x-\lambda t/\hbar)}+G e^{-i(\ell x+\lambda t/\hbar)}
        \end{cases}
      $ \\
      \\
      Assuming a plane wave is only incident from the left, set $G=0$: \\
      \\
      $
        \psi(x)=\begin{cases}
          Ae^{ikx}+Be^{-ikx}, ~~~ x \leq 0 \\
          \\
          F e^{i \ell x}, ~~~~~~~~~~~~~~~~ x>0
        \end{cases} \\
      $ \\
      \\
      $
        T=\dfrac{\dfrac{i \hbar}{2m} \left[(F e^{i \ell x})\dfrac{d}{dx}(F^* e^{-i \ell x})-(F^* e^{-i \ell x})\dfrac{d}{dx}(F e^{i \ell x})\right]}{\dfrac{i \hbar}{2m} \left[(Ae^{ikx})\dfrac{d}{dx}(A^*e^{-ikx})-(A^*e^{-ikx})\dfrac{d}{dx}(Ae^{ikx})\right]} \\
        \\
        =\dfrac{i \ell 2FF^*}{ik AA^*}=\dfrac{\ell}{k}\dfrac{|F|^2}{|A|^2} \\
        \\
        \\
        \therefore ~~~ T=\sqrt{\dfrac{\lambda-V_0}{\lambda}}\dfrac{|F|^2}{|A|^2} ~~ \surd
        \\
        \\
        \\
        R=\dfrac{\dfrac{i \hbar}{2m} \left[(Be^{-ikx})\dfrac{d}{dx}(B^*e^{ikx})-(B^*e^{ikx})\dfrac{d}{dx}(Be^{-ikx})\right]}{\dfrac{i \hbar}{2m} \left[(Ae^{ikx})\dfrac{d}{dx}(A^*e^{-ikx})-(A^*e^{-ikx})\dfrac{d}{dx}(Ae^{ikx})\right]} \\
        \\
        =\dfrac{ik2BB^*}{-ik2AA^*} \\
        \\
        \\
        \\
        \therefore ~~~ R=\dfrac{|B|^2}{|A|^2} ~~ \surd \\
        \\
      $ \\
      Constants, require require the wave function to be continuous at $x=0$. \\ \\
      $
        \lim\limits_{x\to 0^+} \psi(x)=\lim\limits_{x\to 0^-} \psi(x): ~~ A+B=F \\
        \\
        \\
        \bigints_{-\epsilon}^{+\epsilon} \dfrac{d^2 \psi}{dx^2}dx=\bigints_{-\epsilon}^{+\epsilon}  \dfrac{2m}{\hbar^2} \psi(x) \left[V(x)-\lambda\right]dx \\
        \\
        \\
        \dfrac{d \psi}{dx} \Big|_{-\epsilon}^{+\epsilon}=-\bigints_{-\epsilon}^{0} \dfrac{2m \lambda \psi(x)}{\hbar^2}dx+\bigints_{0}^{+\epsilon} \dfrac{2m \psi(x)(V_0-\lambda)}{\hbar^2} dx \\ 
        \\
        \\
        =-\dfrac{2m \lambda \psi(0) \epsilon}{\hbar^2}+\dfrac{2m \psi(0)(V_0-\lambda) \epsilon}{\hbar^2} \\
        \\
        \\
      $
      Taking the limit as $\epsilon \to 0$, we have: $\dfrac{d \psi}{dx}\Big|_{0^-}^{0^+}=0$ \\
      \\
      \\
      $
        \lim\limits_{x\to 0^+} \dfrac{d\psi}{dx}=\lim\limits_{x\to 0^-} \dfrac{d\psi}{dx}: ~~~ ik(A-B)=i\ell f
      $ \\
      \\
      Alrighty, so far we have found these: \\ 
      \\
      $
        \begin{cases}
          A+B=F \\
          \\
          A-B=\dfrac{\ell F}{k}
        \end{cases} \Longrightarrow \begin{cases}
          B=\dfrac{k-\ell}{2k}F \\
          \\
          F=\dfrac{2k}{k+\ell}A
        \end{cases}
      $ \\
      \\
      Now, let's rewrite the $T$ as: \\
      \\
      $
        T=\sqrt{\dfrac{\lambda-V_0}{\lambda}}\dfrac{|F|^2}{|A|^2}=\sqrt{\dfrac{\lambda-V_0}{\lambda}} \left(\dfrac{F}{A}\right) \left(\dfrac{F}{A}\right)^*
        =\sqrt{\dfrac{\lambda-V_0}{\lambda}} \left(\dfrac{2k}{k+\ell}\right) \left(\dfrac{2k}{k+\ell}\right)^* \\
        \\
        \\
        =4\sqrt{\dfrac{\lambda-V_0}{\lambda}} \left(\dfrac{1}{1+\dfrac{\ell}{k}}\right) \left(\dfrac{1}{1+\dfrac{\ell}{k}}\right) \\
        \\
        \\
        =\dfrac{4\sqrt{\dfrac{\lambda-V_0}{\lambda}}}{1+2\sqrt{\dfrac{\lambda-V_0}{\lambda}}+\dfrac{\lambda-V_0}{\lambda}} \\
        \\
        \\
        \\
        \\
        B=\dfrac{k-\ell}{2k}F=\dfrac{k-\ell}{k+\ell}A \\
        \\
        \\
        \Longrightarrow R=\dfrac{|B|^2}{|A|^2}=\left(\dfrac{k-\ell}{k+\ell}\right) \left(\dfrac{k-\ell}{k+\ell}\right)^*=\left(\dfrac{1-\dfrac{\ell}{k}}{1+\dfrac{\ell}{k}}\right) \left(\dfrac{1-\dfrac{\ell}{k}}{1+\dfrac{\ell}{k}}\right) \\
        \\
        \\
        \\
        \\
        R+T=\left(\dfrac{1-\dfrac{\ell}{k}}{1+\dfrac{\ell}{k}}\right) \left(\dfrac{1-\dfrac{\ell}{k}}{1+\dfrac{\ell}{k}}\right) + \dfrac{4\sqrt{\dfrac{\lambda-V_0}{\lambda}}}{1+2\sqrt{\dfrac{\lambda-V_0}{\lambda}}+\dfrac{\lambda-V_0}{\lambda}} \\ \\
      $
      By doing some algebra we can see that $R+T=1 ~~~~ \surd$
      \\
      \\
    }

  \rule{15cm}{1pt}

  \textbf{2.53 (part a only)}
  \textbf{The Scattering Matrix.} The theory of scattering generalizes in a pretty obvious way to
  arbitrary localized potentials (Figure 2.21). To the left (Region I), $V(x)=0$, so 
  $$\psi(x)=A e^{ikx}+B e^{-ikx}$$
  where $k\equiv \dfrac{\sqrt{2mE}}{\hbar}$
  \\
  \\
  To the right (Region III), V (x) is again zero, so 
  $$\psi(x)=F e^{ikx}+G e^{-ikx}$$ 
  In between (Region II), of course, I can’t tell you what $\psi$ is until you specify the potential, but
  because the Schr$\ddot{o}$dinger equation is a linear, second-order differential equation, the general
  solution has got to be of the form 
  $$\psi(x)=C f(x)+D g(x)$$,
  where f(x) and g(x) are two linearly independent particular solutions. There will be four
  boundary conditions (two joining Regions I and II, and two joining Regions II and III). Two of
  these can be used to eliminate C and D, and the other two can be “solved” for B and F in terms
  of A and G:
  $$B=S_{11} A+S_{12} G, ~~~ F=S_{21}A +S_{22} G.$$

  The four coefficients Sij , which depend on k (and hence on E), constitute a $2 \times 2$ matrix \textbf{S}, called
  the \textbf{scattering matrix} (or \textbf{S-matrix} for short). The S-matrix tells you the outgoing
  amplitudes (B and F) in terms of the incoming amplitudes (A and G):
  $$\begin{vmatrix}
    B \\
    F
  \end{vmatrix}=\begin{vmatrix}
    S_{11} & S_{12} \\
    S_{21} & S_{22}
  \end{vmatrix} \begin{vmatrix}
    A \\
    G
  \end{vmatrix}.$$
  In the typical case of scattering from the left, G = 0, so the reflection and transmission
  coefficients are 
  $$R_{\ell}=\dfrac{|B|^2}{|A|^2}\Big|=|S_{11}|^2, ~~~~~ T_{\ell}=\dfrac{|F|^2}{|A|^2}\Big|=|S_{21}|^2.$$

  For scattering from the right, $A=0$, and 
  $$R_r=\dfrac{|F|^2}{|G|^2}\Big|=|S_{22}|^2, ~~~~~ T_r=\dfrac{|B|^2}{|G|^2}\Big|=|S_{12}|^2.$$

  \begin{itemize}
    \item Construct the S-matrix for scattering from a delta-function well (Equation 2.117).

      \textcolor{hwColor}{
        Let's start again with the Schr$\ddot{o}$dinger equation. \\
        \\
        \\
        $i\hbar \dfrac{\partial \Psi}{\partial t}=-\dfrac{\hbar^2}{2m}\dfrac{\partial^2 \Psi}{\partial x}+V \Psi$ \\
        \\
        \\
        When $V(x)=-\alpha \delta(x)$ then the Schr$\ddot{o}$dinger equation becomes: \\
        \\
        $
          i\hbar \dfrac{\partial \Psi}{\partial t}=-\dfrac{\hbar^2}{2m}\dfrac{\partial^2 \Psi}{\partial x}-\alpha \delta(x) \Psi(x,t), -\infty<x<+\infty, ~~~ t>0
        $ \\
        \\
        By the help of the method of separation of variables, we assume $\Psi(x,t)=\psi(x) \phi(t)$ is a solution, hence: \\
        \\
        \\
        $
          \begin{cases}
            E=i\hbar \dfrac{\phi^'(t)}{\phi(t)} \\
            \\
            E=-\dfrac{\hbar^2}{2m}\dfrac{\psi^{''}(x)}{\psi(x)}-\alpha \delta(x)
          \end{cases} \\
          \\
          \\
          \dfrac{d^2 \psi}{dx^2}=-\dfrac{2m}{\hbar^2} \psi(x)\left[\alpha \delta(x)+E\right] \\ \\
        $
        Based on the definition of the delta function we have the following: \\
        \\
        \\
        $
          \dfrac{d^2 \psi}{dx^2}=-\dfrac{2mE \psi(x)}{\hbar^2}, ~~~ x\neq 0
        $ \\
        \\
        The general solution for scattering states is: \\
        \\
        $
          \psi(x)=\begin{cases}
            Ae^{ikx}+Be^{-ikx}, ~~~ x<0 \\
            \\
            Fe^{ikx}+Ge^{-ikx}, ~~~ x>0 \\
          \end{cases} ~~~~ k=\dfrac{\sqrt{2mE}}{\hbar}. \\
        $ \\
        \\
        $
          \lim\limits_{x\to 0^+} \psi(x)=\lim\limits_{x\to 0^-} \psi(x): ~~ A+B=F+G \\
          \\
          \\
          \rule{15cm}{1pt}
          \\
          \\
          \bigints_{-\epsilon}^{+\epsilon} \dfrac{d^2 \psi}{dx^2}dx=\bigints_{-\epsilon}^{+\epsilon}  \dfrac{2m}{\hbar^2} \psi(x) \left[\alpha \delta(x)+E\right]dx \\
          \\
          \dfrac{d \psi}{dx} \Big|_{-\epsilon}^{+\epsilon}=-\dfrac{2m}{\hbar^2} \left[\alpha \bigints_{-\epsilon}^{+\epsilon} \delta(x) \psi(x) dx+E\bigints_{-\epsilon}^{+\epsilon} \psi(x) dx\right]
          =-\dfrac{2m \psi(0)}{\hbar^2} \left[\alpha+2E \epsilon\right] \\
          \\
          \\
        $
        Taking the limit as $\epsilon \to 0$:
        $
          \dfrac{d \psi}{dx} \Big|_{0^-}^{0^+}=-\dfrac{2m \alpha \psi(0)}{\hbar^2} \\
          \\
          \\
          \\
          \lim\limits_{x\to 0^+} \dfrac{d\psi}{dx}-\lim\limits_{x\to 0^-} \dfrac{d\psi}{dx}=-\dfrac{2m \alpha \psi(0)}{\hbar^2}: ~~~~ ik(F-G)-ik(A-B)=-\dfrac{2m \alpha}{\hbar^2}(A+B) \\
          \\
          \\
          \\
          \begin{cases}
            B=\dfrac{im \alpha A}{\hbar^2 k-im\alpha}+\dfrac{\hbar^2 kG}{\hbar^2 k-im\alpha}=S_{11}A+S_{12}G \\
            \\
            \\
            F=\dfrac{im \alpha A}{\hbar^2 k-im\alpha}+\dfrac{\hbar^2 kG}{\hbar^2 k-im\alpha}=S_{21}A+S_{22}G
          \end{cases} \\
          \\
        $
        \\
        \\
        The scattering matrix for the delta-function:
        $
          S=\begin{pmatrix}
            \dfrac{im \alpha}{\hbar^2 k-im \alpha} & \dfrac{\hbar^2 k}{\hbar^2 k-im \alpha} \\
            \\
            \dfrac{\hbar^2 k}{\hbar^2 k-im \alpha} & \dfrac{im \alpha}{\hbar^2 k-im \alpha}
          \end{pmatrix}
        $
      }

  \end{itemize}

\end{document}
