\documentclass[fleqn]{article}
\oddsidemargin 0.0in
\textwidth 6.0in
\thispagestyle{empty}
\usepackage{import}
\usepackage{amsmath}
\usepackage{amssymb}
\usepackage{graphicx}
\usepackage[english]{babel}
\usepackage[utf8x]{inputenc}
\usepackage{float}
\usepackage[colorinlistoftodos]{todonotes}

\definecolor{hwColor}{HTML}{AD53BA}

\begin{document}

  \begin{titlepage}

    \newcommand{\HRule}{\rule{\linewidth}{0.5mm}} % Defines a new command for the horizontal lines, change thickness here

    \center % Center everything on the page



    \textsc{\LARGE Arizona State University}\\[1.5cm] % Name of your university/college

    \textsc{\LARGE Quantum Physics I }\\[1.5cm] % Major heading such as course name


    \begin{figure}
      \includegraphics[width=\linewidth]{asu.png}
    \end{figure}


    \HRule \\[0.4cm]
    { \huge \bfseries Homework Two}\\[0.4cm] 
    \HRule \\[1.5cm]

    \textbf{Behnam Amiri}

    \bigbreak

    \textbf{Prof: Richard Kirian}

    \bigbreak


    \textbf{{\large \today}\\[2cm]}

    \vfill

  \end{titlepage}

  \begin{enumerate}
    \item $(\mathbf{1.11})$ Light of wavelength $350 ~ nm$ is incident on a metallic surface of work function $1.9 ~ eV$.
      \begin{enumerate}
        \item Calculate the kinetic energy of the ejected electrons.

          \textcolor{hwColor}{
            $
              KE=hv-W \\
              \\
              v=\dfrac{c}{\lambda}=\dfrac{3 \times 10^8}{350 \times 10^{-9}} \rightarrow v=8.57 \times 10^{14} ~ Hz \\
              \\
              KE=\left(6.62 \times 10^{-34} ~ J.s\right)\left(8.57 \times 10^{14} ~ Hz\right)-1.9 ~ eV\left(\dfrac{1.6 \times 10^{-19} ~ J}{1 ~ eV}\right) \\
              \\
              KE=2.63 \times 10^{-19} ~ J
            $
          }

        \item Calculate the cutoff frequency of the metal.

          \textcolor{hwColor}{
            $
              W=h.v_0 \Rightarrow v_0=\dfrac{W}{h}=\dfrac{\left(\dfrac{1.6 \times 10^{-19} ~ J}{1 ~ eV} \times 1.9 ~ eV\right)}{6.62 \times 10^{-34} ~ Js} \\
              \\
              v_0=4.59 \times 10^{14} ~ J
            $
          }

      \end{enumerate}

    \item $(\mathbf{1.15})$ If the stopping potential of a metal when illuminated with a radiation of wavelength $150 ~ nm$ is $7.5 ~ V$, 
    calculate the stopping potential of the metal when illuminated by a radiation of wavelength $275 ~ nm$.

      \textcolor{hwColor}{
        $
          V_s=\dfrac{h}{e}v-\dfrac{W}{e}=\dfrac{hc}{\lambda e}-\dfrac{W}{e} \\
          \\
          W=\dfrac{6.62(e-34)}{150(e-9)}c-7.5e \Rightarrow w=1.267 \times 10^{-19} ~ eV \\
          \\
          \\
          V_{s2}=\dfrac{hc}{\lambda_{275 ~ nm}e}-\dfrac{W}{e} \Rightarrow V_{s2}=3.746 ~ v
        $
      }

    \item $(\mathbf{1.20})$ Using energy and momentum conservation requirements, show that a free electron cannot absorb all the energy of a photon.

      \textcolor{hwColor}{
        $
          p^{'}_e=p_e+p_0=\dfrac{hv}{c} \\
          \\
          E_e=E^{'}_e+E_p=m_0c^2+hv \\
          \\
          \therefore \sqrt{m_0c^4+(p^{'}_e c)^2}=m_0c^2+hr \\
          \\
          m^2_0c^4+(hv)^2=m^2_0c^4+(hv)^2+2m_0c^2hv \\ \\
        $
        This is NOT a valid expression!
      }

    \item $(\mathbf{1.22})$ X-rays of wavelength $0.0008 ~ nm$ collide with electrons initially at rest. If the wavelength of the scattered photons is 
    $0.0017 ~ nm$, determine
      \begin{enumerate}
        \item The kinetic energy of the recoiling electrons,

          \textcolor{hwColor}{
            $
              KE_{e^{-}}=hc(\dfrac{1}{\lambda_0}-\dfrac{1}{\lambda_1})=\left(6.62 \times 10^{-34} ~ Js\right)\left(3 \times 10^8\right)\left(\dfrac{1}{0.0008 \times 10^{-9} ~ m}-\dfrac{1}{0.0017 \times 10^{-9} ~ m}\right) \\
              \\
              KE=1.32 \times 10^{-13} ~ J
            $
          }

        \item The angle at which the photons scatter, and

          \textcolor{hwColor}{
            $
              \lambda_f-\lambda_0=\dfrac{h}{mc}(1-cos(\theta)) \Rightarrow cos(\theta)=1-\dfrac{mc(\lambda_f-\lambda_0)}{h} \\
              \\
              \theta=cos^{-1}\left(1-\dfrac{mc(0.0017 \times 10^{-9} ~ m-0.008 \times 10^{-9} ~ m)(3 \times 10^8 ~ m/s)(9.11 \times 10^-{31} ~ kg)}{6.62 \times 10^{-34} ~ Js}\right) \\
              \\
              \theta \approx 48^{\circ}
            $
          }

        \item The angle at which the electrons recoil.

          \textcolor{hwColor}{
            $
              p_{photon}=\dfrac{h}{\lambda}=\dfrac{6.62 \times 10^{-34} ~ Js}{0.0017 \times 10^{-9} m}=3.90 \times 10^{-22}=3.89 \times 10^{-22} \\
              \\
              p=mv \rightarrow p=m\sqrt{\dfrac{2KE}{m}}=9.11 \times 10^{-31} \sqrt{\dfrac{1.315 \times 10^{-13} Js}{9.11 \times 10^{-31} ~ kg}} \\
              \\
              p=4.90 \times 10^{-22} ~ kgm/s \\
              \\
              p_{electron}=p \Rightarrow 4.90 \times 10^{-22} ~ sin(\theta)=3.89 \times 10^{-22} ~ sin(48) \\
              \\
              \Longrightarrow \theta \approx 36.24^{\circ}
            $
          }

      \end{enumerate}

    \item $(\mathbf{1.33})$ Calculate the de Broglie wavelength of
      \begin{enumerate}
        \item An electron of kinetic energy $54 ~ eV$,

          \textcolor{hwColor}{
            $
              \lambda=\dfrac{2\pi \hbar c}{\sqrt{2Tm_e c^2}}=5.574 \times 10^{-19} ~ m
            $
          }

        \item A proton of kinetic energy $70 ~ MeV$,
        
          \textcolor{hwColor}{
            $
            \lambda=\dfrac{2\pi \hbar c}{\sqrt{2Tm_e c^2}}=1.14 \times 10^{-23} ~ m
            $
          }

        \item A $100 ~ g$ bullet moving at $1200 ~ ms^{-1}$, and
         
          \textcolor{hwColor}{
            $
              \lambda=\dfrac{h}{mv}=\dfrac{6.62 \times 10^{-34} }{(0.1)(1200)}=5.52 \times 10^{-36} ~ m
            $
          } \\


        Useful data: $m_ec^2=0.511 ~ MeV, ~ m_pc^2=938.3 ~ MeV, ~ \hbar c \simeq 197.3 ~ eV ~ nm$.
      \end{enumerate}

  \end{enumerate}
\end{document}
