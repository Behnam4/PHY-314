\documentclass[fleqn]{article}
\oddsidemargin 0.0in
\textwidth 6.0in
\thispagestyle{empty}
\usepackage{import}
\usepackage{amsmath}
\usepackage{graphicx}
\usepackage{flexisym}
\usepackage{amssymb}
\usepackage{bigints} 
\usepackage[english]{babel}
\usepackage[utf8x]{inputenc}
\usepackage{float}
\usepackage[colorinlistoftodos]{todonotes}

\definecolor{hwColor}{HTML}{AD53BA}

\begin{document}

  \begin{titlepage}

    \newcommand{\HRule}{\rule{\linewidth}{0.5mm}}

    \center


    \textsc{\LARGE Arizona State University}\\[1.5cm]

    \textsc{\LARGE Quantum Physics I }\\[1.5cm]


    \begin{figure}
      \includegraphics[width=\linewidth]{asu.png}
    \end{figure}


    \HRule \\[0.4cm]
    { \huge \bfseries Homework Seven}\\[0.4cm] 
    \HRule \\[1.5cm]

    \textbf{Behnam Amiri}

    \bigbreak

    \textbf{Prof: Richard Kirian}

    \bigbreak


    \textbf{{\large \today}\\[2cm]}

    \vfill 

  \end{titlepage}

  \textbf{1.14} \\ \\
  Let $P_{ab}(t)$ be the probability of finding the particle in the range $(a<x<b)$, at time $t$.
  \begin{itemize}
    \item Show that $$\dfrac{dP_{ab}}{dt}=J(a,t)-J(b,t)$$ \\
    , where \\
    $$J(x,t)\equiv \dfrac{i \hbar}{2m} \left(\Psi \dfrac{\partial \Psi^*}{\partial x}-\Psi^* \dfrac{\partial \Psi}{\partial x}\right)$$
    What are the units of $J(x,t)?$ Comment: $J$ is called the \textbf{probability current}, because it tells you the rate at which 
    probability is "flowing" past the point $x$. If $P_{ab}(t)$ is increasing, then more probability is flowing into the region
    at one end than flows out at the other.


      \textcolor{hwColor}{
        In the simplest form,Born’s interpretation states that the probability density of finding a particle at a given point, 
        when measured, is proportional to the square of the magnitude of the particle's wavefunction at that point. \\
        \\
        $$
          P_{ab}(t)= \bigints_{a}^{b} |\Psi(x,t)|^2 dx=\bigints_{a}^{b} \Psi(x,t) \Psi^*(x,t) dx \\ \\
        $$
        $
          \dfrac{dP_{ab}}{dt}=\dfrac{d}{dt} \left(\bigints_{a}^{b} \Psi(x,t) \Psi^*(x,t) dx\right)=\bigints_{a}^{b} \dfrac{\partial}{\partial t} \left(\Psi(x,t) \Psi^*(x,t)\right) dx \\
          \\
          =\bigints_{a}^{b} \left[\dfrac{\partial \Psi}{\partial t}\Psi^*+\dfrac{\partial \Psi^*}{\partial t}\Psi\right] dx \\ \\
        $
        From the textbook we have the Schr$\ddot{o}$dinger equation as: \\
        \\
        $
          i\hbar \dfrac{\partial \Psi}{\partial t}=-\dfrac{\hbar^2}{2m}\dfrac{\partial^2 \Psi}{\partial x^2}+V \Psi
          \Rightarrow \dfrac{\partial \Psi}{\partial t}=-\dfrac{\hbar}{2mi}\dfrac{\partial^2 \Psi}{\partial x^2}+\dfrac{1}{i\hbar}V \Psi \\
        $
        \\
        \\
        Let's multiple both sides by $\dfrac{i}{i}$, then we get: \\
        \\
        $
          \dfrac{\partial \Psi}{\partial t}=\dfrac{\hbar i}{2m} \dfrac{\partial^2 \Psi}{\partial x^2}-\dfrac{i}{\hbar} V \Psi 
          \Rightarrow \dfrac{\partial \Psi^*}{\partial t}=-\dfrac{\hbar i}{2m} \dfrac{\partial^2 \Psi^*}{\partial x^2}+\dfrac{i}{\hbar} V \Psi^* \\ \\
          \\
        $
        Now, we are ready to evaluate $ \dfrac{dP_{ab}}{dt}$; \\
        \\
        $
          \dfrac{dP_{ab}}{dt}=\bigints_{a}^{b} \left[\left(\dfrac{\hbar i}{2m} \dfrac{\partial^2 \Psi}{\partial x^2}-\dfrac{i}{\hbar} V \Psi\right) \Psi^*+\left(-\dfrac{\hbar i}{2m} \dfrac{\partial^2 \Psi^*}{\partial x^2}+\dfrac{i}{\hbar} V \Psi^*\right)\Psi\right] dx=\bigints_{a}^{b} \dfrac{\hbar i}{2m} \left[\dfrac{\partial^2 \psi}{\partial x^2}\Psi^*-\dfrac{\partial \Psi^*}{\partial x^2}\Psi\right]dx \\
          \\
          \\
          =\dfrac{\hbar i}{2m} \bigints_{a}^{b} \left[\left(\dfrac{\partial \Psi^*}{\partial x}\dfrac{\partial \Psi}{\partial x}+\dfrac{\partial^2 \Psi}{\partial x^2} \Psi^*\right)-\left(\Psi \dfrac{\partial^2 \Psi^*}{\partial x^2}+\dfrac{\partial \Psi^*}{\partial x} \dfrac{\partial \Psi}{\partial x}\right)\right]dx \\
          \\
          \\
          =\dfrac{\hbar i}{2m} \bigints_{a}^{b} \dfrac{\partial}{\partial x} \left[\dfrac{\partial \Psi}{\partial x} \Psi^* -\Psi \dfrac{\partial \Psi^*}{\partial x}\right]dx \\
          \\
          \\
          =\dfrac{\hbar i}{2m} \left[\left(\Psi^*(b,t) \dfrac{\partial \Psi}{\partial x}(b,t)-\Psi(b,t) \dfrac{\Psi^*}{\partial x}(b,t)\right)-\left(\Psi^*(a,t) \dfrac{\partial \Psi}{\partial x}(a,t)-\Psi(a,t) \dfrac{\Psi^*}{\partial x}(a,t)\right)\right] \\
          \\
          \\
          \\
          \therefore ~~~ \dfrac{dP_{ab}}{dt}=\mathcal{J}(b,t)-\mathcal{J}(a,t) \\ \\
        $
        $P_{ab}$ is dimensionless, therefore $\mathcal{J}$ has $\dfrac{1}{s}$ as its unit. \\
        \\
      }

    \item Find the probability current for the wave function in Problem 1.9. (This is not a very pithy example, I am afraid
    ; we will encounter more substantial ones in due course.)

      \textcolor{hwColor}{
        $
          J(x,t)=\dfrac{i \hbar}{2m} \left(\Psi \dfrac{\partial \Psi^*}{\partial x}-\Psi^* \dfrac{\partial \Psi}{\partial x}\right) \\
          \\
          \\
          =\dfrac{i \hbar}{2m}\left(\sqrt[4]{\dfrac{2am}{\pi \hbar}}e^{-a\left[(mx^2/\hbar)+it\right]}\right) \dfrac{\partial}{\partial x} \left(\sqrt[4]{\dfrac{2am}{\pi \hbar}}e^{-a\left[(mx^2/\hbar)-it\right]}\right) \\ \\
          -\dfrac{i \hbar}{2m}\left(\sqrt[4]{\dfrac{2am}{\pi \hbar}}e^{-a\left[(mx^2/\hbar)+it\right]}\right) \dfrac{\partial}{\partial x} \left(\sqrt[4]{\dfrac{2am}{\pi \hbar}}e^{-a\left[(mx^2/\hbar)-it\right]}\right)
          \\
          \\
          \\
          \therefore ~~~  J(x,t)=0
        $
      }

  \end{itemize}

  \rule{15cm}{1pt}

  \textbf{2.18}
  \begin{itemize}
    \item Find the probability current, J (Problem 1.14) for the free particlen wave function Equation 2.95. which
    direction does the probability flow?

      \textcolor{hwColor}{
        Let's rewrite equation $(2.95)$ from the textbook: \\
        \\
        $$\Psi_k(x,t)=A e^{i(kx-\dfrac{\hbar k^2}{2m}t)}$$ \\
        \\
        From problem $(1.14)$ we found $J(x,t)$. Let's calculate the probability current: \\
        \\
        \\
        $
          J(x,t)=\dfrac{\hbar i}{2m} \left[\Psi \dfrac{\partial \Psi^*}{\partial x}-\dfrac{\partial \Psi}{\partial x}\Psi^*\right] \\
          \\
          \\
          =\dfrac{\hbar i}{2m} A e^{i(kx-\dfrac{\hbar k^2}{2m}t)} \dfrac{\partial }{\partial x} \left(A^* e^{i(-kx+\dfrac{\hbar k^2}{2m}t)}\right) \\
          -\dfrac{\hbar i}{2m} A^* e^{i(-kx+\dfrac{\hbar k^2}{2m}t)} \dfrac{\partial }{\partial x} \left(A e^{i(kx-\dfrac{\hbar k^2}{2m}t)}\right) \\
          \\
          \\
          =\dfrac{\hbar i |A|^2}{2m} \left(-ik-ik\right) \\
          \\
          \\
          \\
          \therefore ~~~  J(x,t)=\dfrac{\hbar k |A|^2}{m} \\ \\
        $
        Based on what we found, the direction of the probability flow depends on the sign of $k$. When $k$ is positive, 
        the probability flow is in the x-direction. \\
        \\
        When $k$ is positive, the probability flow is in the negative x-direction. 
      }

  \end{itemize}

  \rule{15cm}{1pt}

  \textbf{2.20} \\ \\
  A free particle has the initial wave function \\
  $$\Psi(x,0)=Ae^{-a|x|}$$
  where $A$ and $a$ are positive real constants.
  \begin{itemize}
    \item Normalize $\Psi(x,0)$.

    \textcolor{hwColor}{
      We need to start with the Schr$\ddot{o}$dinger equation: \\
      \\
      $
        i\hbar \dfrac{\partial \Psi}{\partial t}=-\dfrac{\hbar^2}{2m}\dfrac{\partial^2 \Psi}{\partial x^2}+V \Psi \\ \\
      $
      We know that for a free particle $V=0$, therfore we are left with: \\
      \\
      $
        i\hbar \dfrac{\partial \Psi}{\partial t}=-\dfrac{\hbar^2}{2m}\dfrac{\partial^2 \Psi}{\partial x^2} \\ \\
      $
      At this point we need to take the Fourier transform of both sides of the Schr$\ddot{o}$dinger equation: \\
      \\
      \\
      $
        \mathcal{F} \left(i\hbar \dfrac{\partial \Psi}{\partial t}\right)=\mathcal{F} \left(-\dfrac{\hbar^2}{2m}\dfrac{\partial^2 \Psi}{\partial x^2}\right) \\
        \\
        \\
        i\hbar \mathcal{F} \left( \dfrac{\partial \Psi}{\partial t}\right)=-\dfrac{\hbar^2}{2m}\mathcal{F} \left(\dfrac{\partial^2 \Psi}{\partial x^2}\right) 
        \Rightarrow i\hbar \dfrac{d \tilde{\Psi}}{dt}=-\dfrac{\hbar^2}{2m} (ik)^2 \tilde{\Psi}(k,t) \\
        \\
      $
      We made our life easier. Why? Because now we have a first order differential equation which we can easily 
      find its solution: \\
      \\
      $
        i\dfrac{\dfrac{d \tilde{\Psi}(k,t)}{dt}}{\tilde{\Psi}(k,t)}=\dfrac{\hbar k^2}{2m} \Rightarrow Ln \tilde{\Psi}(k,t)=-\dfrac{i \hbar}{2m}k^2 t \\
        \\
        \\
        \therefore ~~~ \tilde{\Psi}(k,t)=\tilde{\Psi}_0(k) e^{-\dfrac{i \hbar}{2m}k^2 t} \\ \\
      $
      We are interested in $\Psi(k,t)$, hence by taking the inverse Fourier transform we have: \\
      \\
      \\
      $
        \Psi(x,t)=\mathcal{F}^{-1} \left[\tilde{\Psi}(k,t)\right]=\mathcal{F}^{-1} \left[\tilde{\Psi}_0(k) e^{-\dfrac{i \hbar}{2m}k^2 t}\right]=\dfrac{1}{\sqrt{2\pi}} \bigints_{-\infty}^{+\infty} e^{ikx} \tilde{\Psi}_0(k) e^{-\dfrac{i \hbar}{2m}k^2 t} dk \\
        \\
        \\
        \therefore ~~~ \Psi(x,t)=\dfrac{1}{\sqrt{2\pi}}  \bigints_{-\infty}^{+\infty} e^{ikx-\dfrac{\hbar ik^2 t}{2m}} \tilde{\Psi}_0(k) e^{-\dfrac{i \hbar}{2m}k^2 t} dk
      $ 
    }

    \item Find $\phi(k)$.

      \textcolor{hwColor}{
        $\phi(k)$ is the Fourier transform of the initial wave function. \\
        \\
        $$
          \phi(k)=\tilde{\Psi}_0(k) \\ \\
        $$
        In order to find $\tilde{\Psi}_0(k)$ we have to find the normalized of the initial wave function. \\
        \\
        \\
        $
          \bigints_{-\infty}^{+\infty} |\Psi(x,0)|^2 dx=\bigints_{-\infty}^{+\infty} A^2 e^{-2ax} dx=2A^2 \bigints_{0}^{+\infty} e^{-2ax} dx=1 \\
          \\
          \\
          \therefore ~~~ \dfrac{A^2}{a}=1 \Longrightarrow A=\sqrt{a} \Longrightarrow \Psi_0(x)=\sqrt{a} e^{-ax} \\
        $
        It's time to find $\phi(k)$; \\
        \\
        \\
        $
          \phi(k)=\tilde{\Psi}_0(k)=\mathcal{F} \left[\Psi_0(x)\right]=\mathcal{F} \left[\sqrt{a} e^{-ax}\right]=\dfrac{1}{\sqrt{2 \pi}} \bigints_{-\infty}^{+\infty} e^{-ikx} \sqrt{a} e^{-ax} dx \\
          \\
          \\
          =\sqrt{\dfrac{a}{2 \pi}} \left[\bigints_{-\infty}^{0} e^{-ikx+ax} dx+ \bigints_{0}^{+\infty} e^{-ikx-ax} dx\right] \\
          \\
          \\
          \therefore ~~~ \phi(k)=\sqrt{\dfrac{a}{2\pi}} \dfrac{2a}{k^2+a^2}
        $
      }

    \item Construct $\Psi(x,t)$, in the form of an integral.
    
      \textcolor{hwColor}{
        $
          \Psi(x,t)=\dfrac{1}{\sqrt{2\pi}} \bigints_{-\infty}^{+\infty} exp\left(ikx-\dfrac{\hbar ik^2t}{2m}\right) \tilde{\Psi}_0(k) dk \\
          \\
          =\dfrac{\sqrt{a^3}}{\pi} \bigints_{-\infty}^{+\infty} \dfrac{1}{k^2+a^2}exp\left(ikx-\dfrac{\hbar ik^2t}{2m}\right) \tilde{\Psi}_0(k) dk
        $
      }

    \item Discuss the limiting cases (a very large, and a very small).

      \textcolor{hwColor}{
        \[\lim_{x\to 0} \Psi(x,t)=\lim_{x\to 0} \dfrac{\sqrt{a^3}}{\pi} \bigints_{-\infty}^{+\infty} \dfrac{1}{k^2+a^2} exp\left(ikx-\dfrac{\hbar ik^2 t}{2m}\right) dk\] \\
        \[=\dfrac{1}{\pi} \bigints_{-\infty}^{+\infty} exp\left(ikx-\dfrac{\hbar ik^2 t}{2m}\right) \lim_{x\to 0} \left(\dfrac{\sqrt{a^3}}{k^2+a^2}\right) dk\] \\
        \[=\dfrac{1}{\pi} \bigints_{-\infty}^{+\infty} exp\left(ikx-\dfrac{\hbar ik^2 t}{2m}\right) \times 0 dk=0\] \\
      }

      \rule{15cm}{1pt}

      \textcolor{hwColor}{
        \[\lim_{x\to\infty} \Psi(x,t)=\lim_{x\to\infty} \dfrac{\sqrt{a^3}}{\pi} \bigints_{-\infty}^{+\infty} \dfrac{1}{k^2+a^2} exp\left(ikx-\dfrac{\hbar ik^2 t}{2m}\right) dk\] \\
        \[=\dfrac{1}{\pi} \bigints_{-\infty}^{+\infty} exp\left(ikx-\dfrac{\hbar ik^2 t}{2m}\right)  \lim_{x \to \infty} \dfrac{\sqrt{a^3}}{k^2+a^2}  dk=0\] \\
        \\
        The Fourier transform of $\Psi(x,0)$ is basically the wave function for the particle’s momentum at $t=0$. 
        For low $a$ a low certainty in the particle’s position and a high certainty in the particle’s momentum initially. \\
        \\
        For high $a$, the following graphs indicate a high certainty in the particle’s position and a low certainty in the particle’s momentum initially.
      }

  \end{itemize}

  \rule{15cm}{1pt}

  \textbf{2.45} \\ \\
  In this problem you will show that the number of nodes of the stationary states of a one-dimensional potential always 
  increases with energy. Consider two (real, normalized) solutions ($\psi_n$ and $\psi_m$) to the time-independent 
  Schr$\ddot{o}$dinger equation (for a given potential $V(x)$), with energies $E_n > E_m$.
  \begin{itemize}
    \item Show that \\
    $$\dfrac{d}{dx} \left(\dfrac{d \psi_m}{dx}\psi_n-\psi_m\dfrac{d \psi_n}{dx}\right)=\dfrac{2m}{\hbar^2}(E_n-E_m) \psi_m \psi_n$$

    \item Let $x_1$ and $x_2$ be two adjacent nodes of the function $\psi_m(x)$.Show that \\
    $$\psi^'_m(x_2) \psi_n(x_2)-\psi^'_m(x_1) \psi_n(x_1)=\dfrac{2m}{\hbar^2}(E_n-E_m) \bigints_{x_1}^{x_2} \psi_m \psi_n dx$$

    \item If $\psi_n(x)$ has no nodes between $x_1$ and $x_2$, then it must have the same sign everywhere 
    in the interval. Show that (b) then leads to a contradiction. Therefore, between every pair of nodes of
    $\psi_m(x), ~ \psi_n(x)$ must have at \emph{least} one node, and in particular the number of nodes 
    increases with energy.
  \end{itemize}

  \rule{15cm}{1pt}

  \textbf{2.46}
  \\
  Imagine a bead of mass $m$ that slides frictionlessly around a circular wire ring of circumference $L$. 
  (This is just like a free particle, except that $\psi(x+L)=\psi(x)$) Find trhe stationary states (with
  appropriate normalization) and the corresponding allowed energies. Note that there are (with one exception)
  \emph{two} independent solutions for each energy $E_n$ corresponding to clockwise and counterf-clockwise
  circulation; call them $\psi^+_n(x)$ and $\psi^-_n(x)$. How do you account for this degeneracy, in
  view of the theorem in Problem 2.44 (why does the theorem fail, in this case)?

\end{document}
