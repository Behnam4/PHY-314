\documentclass[fleqn]{article}
\oddsidemargin 0.0in
\textwidth 6.0in
\thispagestyle{empty}
\usepackage{import}
\usepackage{amsmath}
\usepackage{graphicx}
\usepackage{flexisym}
\usepackage{amssymb}
\usepackage{bigints} 
\usepackage[english]{babel}
\usepackage[utf8x]{inputenc}
\usepackage{float}
\usepackage[colorinlistoftodos]{todonotes}

\definecolor{hwColor}{HTML}{AD53BA}

\begin{document}

  \begin{titlepage}

    \newcommand{\HRule}{\rule{\linewidth}{0.5mm}}

    \center


    \textsc{\LARGE Arizona State University}\\[1.5cm]

    \textsc{\LARGE Quantum Physics I }\\[1.5cm]


    \begin{figure}
      \includegraphics[width=\linewidth]{asu.png}
    \end{figure}


    \HRule \\[0.4cm]
    { \huge \bfseries Homework Seven}\\[0.4cm] 
    \HRule \\[1.5cm]

    \textbf{Behnam Amiri}

    \bigbreak

    \textbf{Prof: Richard Kirian}

    \bigbreak


    \textbf{{\large \today}\\[2cm]}

    \vfill 

  \end{titlepage}

  \textbf{1.14} \\ \\
  Let $P_{ab}(t)$ be the probability of finding the particle in the range $(a<x<b)$, at time $t$.
  \begin{itemize}
    \item Show that $$\dfrac{dP_{ab}}{dt}=J(a,t)-J(b,t)$$ \\
    , where \\
    $$J(x,t)\equiv \dfrac{i \hbar}{2m} \left(\Psi \dfrac{\partial \Psi^*}{\partial x}-\Psi^* \dfrac{\partial \Psi}{\partial x}\right)$$
    What are the units of $J(x,t)?$ Comment: $J$ is called the \textbf{probability current}, because it tells you the rate at which 
    probability is "flowing" past the point $x$. If $P_{ab}(t)$ is increasing, then more probability is flowing into the region
    at one end than flows out at the other.


      \textcolor{hwColor}{
        In the simplest form,Born’s interpretation states that the probability density of finding a particle at a given point, 
        when measured, is proportional to the square of the magnitude of the particle's wavefunction at that point. \\
        \\
        $$
          P_{ab}(t)= \bigints_{a}^{b} |\Psi(x,t)|^2 dx=\bigints_{a}^{b} \Psi(x,t) \Psi^*(x,t) dx \\ \\
        $$
        $
          \dfrac{dP_{ab}}{dt}=\dfrac{d}{dt} \left(\bigints_{a}^{b} \Psi(x,t) \Psi^*(x,t) dx\right)=\bigints_{a}^{b} \dfrac{\partial}{\partial t} \left(\Psi(x,t) \Psi^*(x,t)\right) dx \\
          \\
          =\bigints_{a}^{b} \left[\dfrac{\partial \Psi}{\partial t}\Psi^*+\dfrac{\partial \Psi^*}{\partial t}\Psi\right] dx \\ \\
        $
        From the textbook we have the Schr$\ddot{o}$dinger equation as: \\
        \\
        $
          i\hbar \dfrac{\partial \Psi}{\partial t}=-\dfrac{\hbar^2}{2m}\dfrac{\partial^2 \Psi}{\partial x^2}+V \Psi
          \Rightarrow \dfrac{\partial \Psi}{\partial t}=-\dfrac{\hbar}{2mi}\dfrac{\partial^2 \Psi}{\partial x^2}+\dfrac{1}{i\hbar}V \Psi \\
        $
        \\
        \\
        Let's multiple both sides by $\dfrac{i}{i}$, then we get: \\
        \\
        $
          \dfrac{\partial \Psi}{\partial t}=\dfrac{\hbar i}{2m} \dfrac{\partial^2 \Psi}{\partial x^2}-\dfrac{i}{\hbar} V \Psi 
          \Rightarrow \dfrac{\partial \Psi^*}{\partial t}=-\dfrac{\hbar i}{2m} \dfrac{\partial^2 \Psi^*}{\partial x^2}+\dfrac{i}{\hbar} V \Psi^* \\ \\
          \\
        $
        Now, we are ready to evaluate $ \dfrac{dP_{ab}}{dt}$; \\
        \\
        $
          \dfrac{dP_{ab}}{dt}=\bigints_{a}^{b} \left[\left(\dfrac{\hbar i}{2m} \dfrac{\partial^2 \Psi}{\partial x^2}-\dfrac{i}{\hbar} V \Psi\right) \Psi^*+\left(-\dfrac{\hbar i}{2m} \dfrac{\partial^2 \Psi^*}{\partial x^2}+\dfrac{i}{\hbar} V \Psi^*\right)\Psi\right] dx=\bigints_{a}^{b} \dfrac{\hbar i}{2m} \left[\dfrac{\partial^2 \psi}{\partial x^2}\Psi^*-\dfrac{\partial \Psi^*}{\partial x^2}\Psi\right]dx \\
          \\
          \\
          =\dfrac{\hbar i}{2m} \bigints_{a}^{b} \left[\left(\dfrac{\partial \Psi^*}{\partial x}\dfrac{\partial \Psi}{\partial x}+\dfrac{\partial^2 \Psi}{\partial x^2} \Psi^*\right)-\left(\Psi \dfrac{\partial^2 \Psi^*}{\partial x^2}+\dfrac{\partial \Psi^*}{\partial x} \dfrac{\partial \Psi}{\partial x}\right)\right]dx \\
          \\
          \\
          =\dfrac{\hbar i}{2m} \bigints_{a}^{b} \dfrac{\partial}{\partial x} \left[\dfrac{\partial \Psi}{\partial x} \Psi^* -\Psi \dfrac{\partial \Psi^*}{\partial x}\right]dx \\
          \\
          \\
          =\dfrac{\hbar i}{2m} \left[\left(\Psi^*(b,t) \dfrac{\partial \Psi}{\partial x}(b,t)-\Psi(b,t) \dfrac{\Psi^*}{\partial x}(b,t)\right)-\left(\Psi^*(a,t) \dfrac{\partial \Psi}{\partial x}(a,t)-\Psi(a,t) \dfrac{\Psi^*}{\partial x}(a,t)\right)\right] \\
          \\
          \\
          \\
          \therefore ~~~ \dfrac{dP_{ab}}{dt}=\mathcal{J}(b,t)-\mathcal{J}(a,t) \\ \\
        $
        $P_{ab}$ is dimensionless, therefore $\mathcal{J}$ has $\dfrac{1}{s}$ as its unit. \\
        \\
        \\
        \emph{Note: There is a small error ion my calculation, then end result should be (Missed a negative sign)} \\
        \\
        $\therefore ~~~ \dfrac{dP_{ab}}{dt}=\mathcal{J}(a,t)-\mathcal{J}(b,t) ~~~~ \surd$ 
      }

    \item Find the probability current for the wave function in Problem 1.9. (This is not a very pithy example, I am afraid
    ; we will encounter more substantial ones in due course.)

      \textcolor{hwColor}{
        $
          J(x,t)=\dfrac{i \hbar}{2m} \left(\Psi \dfrac{\partial \Psi^*}{\partial x}-\Psi^* \dfrac{\partial \Psi}{\partial x}\right) \\
          \\
          \\
          =\dfrac{i \hbar}{2m}\left(\sqrt[4]{\dfrac{2am}{\pi \hbar}}e^{-a\left[(mx^2/\hbar)+it\right]}\right) \dfrac{\partial}{\partial x} \left(\sqrt[4]{\dfrac{2am}{\pi \hbar}}e^{-a\left[(mx^2/\hbar)-it\right]}\right) \\ \\
          -\dfrac{i \hbar}{2m}\left(\sqrt[4]{\dfrac{2am}{\pi \hbar}}e^{-a\left[(mx^2/\hbar)+it\right]}\right) \dfrac{\partial}{\partial x} \left(\sqrt[4]{\dfrac{2am}{\pi \hbar}}e^{-a\left[(mx^2/\hbar)-it\right]}\right)
          \\
          \\
          \\
          \therefore ~~~  J(x,t)=0
        $
      }

  \end{itemize}

  \rule{15cm}{1pt}

  \textbf{2.18}
  \begin{itemize}
    \item Find the probability current, J (Problem 1.14) for the free particlen wave function Equation 2.95. which
    direction does the probability flow?

      \textcolor{hwColor}{
        Let's rewrite equation $(2.95)$ from the textbook: \\
        \\
        $$\Psi_k(x,t)=A e^{i(kx-\dfrac{\hbar k^2}{2m}t)}$$ \\
        \\
        From problem $(1.14)$ we found $J(x,t)$. Let's calculate the probability current: \\
        \\
        \\
        $
          J(x,t)=\dfrac{\hbar i}{2m} \left[\Psi \dfrac{\partial \Psi^*}{\partial x}-\dfrac{\partial \Psi}{\partial x}\Psi^*\right] \\
          \\
          \\
          =\dfrac{\hbar i}{2m} A e^{i(kx-\dfrac{\hbar k^2}{2m}t)} \dfrac{\partial }{\partial x} \left(A^* e^{i(-kx+\dfrac{\hbar k^2}{2m}t)}\right) \\
          -\dfrac{\hbar i}{2m} A^* e^{i(-kx+\dfrac{\hbar k^2}{2m}t)} \dfrac{\partial }{\partial x} \left(A e^{i(kx-\dfrac{\hbar k^2}{2m}t)}\right) \\
          \\
          \\
          =\dfrac{\hbar i |A|^2}{2m} \left(-ik-ik\right) \\
          \\
          \\
          \\
          \therefore ~~~  J(x,t)=\dfrac{\hbar k |A|^2}{m} \\ \\
        $
        Based on what we found, the direction of the probability flow depends on the sign of $k$. When $k$ is positive, 
        the probability flow is in the x-direction. \\
        \\
        When $k$ is positive, the probability flow is in the negative x-direction. 
      }

  \end{itemize}

  \rule{15cm}{1pt}

  \textbf{2.20} \\ \\
  A free particle has the initial wave function \\
  $$\Psi(x,0)=Ae^{-a|x|}$$
  where $A$ and $a$ are positive real constants.
  \begin{itemize}
    \item Normalize $\Psi(x,0)$.

    \textcolor{hwColor}{
      We need to start with the Schr$\ddot{o}$dinger equation: \\
      \\
      $
        i\hbar \dfrac{\partial \Psi}{\partial t}=-\dfrac{\hbar^2}{2m}\dfrac{\partial^2 \Psi}{\partial x^2}+V \Psi \\ \\
      $
      We know that for a free particle $V=0$, therfore we are left with: \\
      \\
      $
        i\hbar \dfrac{\partial \Psi}{\partial t}=-\dfrac{\hbar^2}{2m}\dfrac{\partial^2 \Psi}{\partial x^2} \\ \\
      $
      At this point we need to take the Fourier transform of both sides of the Schr$\ddot{o}$dinger equation: \\
      \\
      \\
      $
        \mathcal{F} \left(i\hbar \dfrac{\partial \Psi}{\partial t}\right)=\mathcal{F} \left(-\dfrac{\hbar^2}{2m}\dfrac{\partial^2 \Psi}{\partial x^2}\right) \\
        \\
        \\
        i\hbar \mathcal{F} \left( \dfrac{\partial \Psi}{\partial t}\right)=-\dfrac{\hbar^2}{2m}\mathcal{F} \left(\dfrac{\partial^2 \Psi}{\partial x^2}\right) 
        \Rightarrow i\hbar \dfrac{d \tilde{\Psi}}{dt}=-\dfrac{\hbar^2}{2m} (ik)^2 \tilde{\Psi}(k,t) \\
        \\
      $
      We made our life easier. Why? Because now we have a first order differential equation which we can easily 
      find its solution: \\
      \\
      $
        i\dfrac{\dfrac{d \tilde{\Psi}(k,t)}{dt}}{\tilde{\Psi}(k,t)}=\dfrac{\hbar k^2}{2m} \Rightarrow Ln \tilde{\Psi}(k,t)=-\dfrac{i \hbar}{2m}k^2 t \\
        \\
        \\
        \therefore ~~~ \tilde{\Psi}(k,t)=\tilde{\Psi}_0(k) e^{-\dfrac{i \hbar}{2m}k^2 t} \\ \\
      $
      We are interested in $\Psi(k,t)$, hence by taking the inverse Fourier transform we have: \\
      \\
      \\
      $
        \Psi(x,t)=\mathcal{F}^{-1} \left[\tilde{\Psi}(k,t)\right]=\mathcal{F}^{-1} \left[\tilde{\Psi}_0(k) e^{-\dfrac{i \hbar}{2m}k^2 t}\right]=\dfrac{1}{\sqrt{2\pi}} \bigints_{-\infty}^{+\infty} e^{ikx} \tilde{\Psi}_0(k) e^{-\dfrac{i \hbar}{2m}k^2 t} dk \\
        \\
        \\
        \therefore ~~~ \Psi(x,t)=\dfrac{1}{\sqrt{2\pi}}  \bigints_{-\infty}^{+\infty} e^{ikx-\dfrac{\hbar ik^2 t}{2m}} \tilde{\Psi}_0(k) e^{-\dfrac{i \hbar}{2m}k^2 t} dk
      $ 
    }

    \item Find $\phi(k)$.

      \textcolor{hwColor}{
        $\phi(k)$ is the Fourier transform of the initial wave function. \\
        \\
        $$
          \phi(k)=\tilde{\Psi}_0(k) \\ \\
        $$
        In order to find $\tilde{\Psi}_0(k)$ we have to find the normalized of the initial wave function. \\
        \\
        \\
        $
          \bigints_{-\infty}^{+\infty} |\Psi(x,0)|^2 dx=\bigints_{-\infty}^{+\infty} A^2 e^{-2ax} dx=2A^2 \bigints_{0}^{+\infty} e^{-2ax} dx=1 \\
          \\
          \\
          \therefore ~~~ \dfrac{A^2}{a}=1 \Longrightarrow A=\sqrt{a} \Longrightarrow \Psi_0(x)=\sqrt{a} e^{-ax} \\
        $
        It's time to find $\phi(k)$; \\
        \\
        \\
        $
          \phi(k)=\tilde{\Psi}_0(k)=\mathcal{F} \left[\Psi_0(x)\right]=\mathcal{F} \left[\sqrt{a} e^{-ax}\right]=\dfrac{1}{\sqrt{2 \pi}} \bigints_{-\infty}^{+\infty} e^{-ikx} \sqrt{a} e^{-ax} dx \\
          \\
          \\
          =\sqrt{\dfrac{a}{2 \pi}} \left[\bigints_{-\infty}^{0} e^{-ikx+ax} dx+ \bigints_{0}^{+\infty} e^{-ikx-ax} dx\right] \\
          \\
          \\
          \therefore ~~~ \phi(k)=\sqrt{\dfrac{a}{2\pi}} \dfrac{2a}{k^2+a^2}
        $
      }

    \item Construct $\Psi(x,t)$, in the form of an integral.
    
      \textcolor{hwColor}{
        $
          \Psi(x,t)=\dfrac{1}{\sqrt{2\pi}} \bigints_{-\infty}^{+\infty} exp\left(ikx-\dfrac{\hbar ik^2t}{2m}\right) \tilde{\Psi}_0(k) dk \\
          \\
          =\dfrac{\sqrt{a^3}}{\pi} \bigints_{-\infty}^{+\infty} \dfrac{1}{k^2+a^2}exp\left(ikx-\dfrac{\hbar ik^2t}{2m}\right) \tilde{\Psi}_0(k) dk
        $
      }

    \item Discuss the limiting cases (a very large, and a very small).

      \textcolor{hwColor}{
        \[\lim_{x\to 0} \Psi(x,t)=\lim_{x\to 0} \dfrac{\sqrt{a^3}}{\pi} \bigints_{-\infty}^{+\infty} \dfrac{1}{k^2+a^2} exp\left(ikx-\dfrac{\hbar ik^2 t}{2m}\right) dk\] \\
        \[=\dfrac{1}{\pi} \bigints_{-\infty}^{+\infty} exp\left(ikx-\dfrac{\hbar ik^2 t}{2m}\right) \lim_{x\to 0} \left(\dfrac{\sqrt{a^3}}{k^2+a^2}\right) dk\] \\
        \[=\dfrac{1}{\pi} \bigints_{-\infty}^{+\infty} exp\left(ikx-\dfrac{\hbar ik^2 t}{2m}\right) \times 0 dk=0\] \\
      }

      \rule{15cm}{1pt}

      \textcolor{hwColor}{
        \[\lim_{x\to\infty} \Psi(x,t)=\lim_{x\to\infty} \dfrac{\sqrt{a^3}}{\pi} \bigints_{-\infty}^{+\infty} \dfrac{1}{k^2+a^2} exp\left(ikx-\dfrac{\hbar ik^2 t}{2m}\right) dk\] \\
        \[=\dfrac{1}{\pi} \bigints_{-\infty}^{+\infty} exp\left(ikx-\dfrac{\hbar ik^2 t}{2m}\right)  \lim_{x \to \infty} \dfrac{\sqrt{a^3}}{k^2+a^2}  dk=0\] \\
        \\
        The Fourier transform of $\Psi(x,0)$ is basically the wave function for the particle’s momentum at $t=0$. 
        For low $a$ a low certainty in the particle’s position and a high certainty in the particle’s momentum initially. \\
        \\
        For high $a$, the following graphs indicate a high certainty in the particle’s position and a low certainty in the particle’s momentum initially.
      }

  \end{itemize}

  \rule{15cm}{1pt}

  \textbf{2.45} \\ \\
  In this problem you will show that the number of nodes of the stationary states of a one-dimensional potential always 
  increases with energy. Consider two (real, normalized) solutions ($\psi_n$ and $\psi_m$) to the time-independent 
  Schr$\ddot{o}$dinger equation (for a given potential $V(x)$), with energies $E_n > E_m$.
  \begin{itemize}
    \item Show that \\
    $$\dfrac{d}{dx} \left(\dfrac{d \psi_m}{dx}\psi_n-\psi_m\dfrac{d \psi_n}{dx}\right)=\dfrac{2m}{\hbar^2}(E_n-E_m) \psi_m \psi_n$$

    \item Let $x_1$ and $x_2$ be two adjacent nodes of the function $\psi_m(x)$.Show that \\
    $$\psi^'_m(x_2) \psi_n(x_2)-\psi^'_m(x_1) \psi_n(x_1)=\dfrac{2m}{\hbar^2}(E_n-E_m) \bigints_{x_1}^{x_2} \psi_m \psi_n dx$$

    \item If $\psi_n(x)$ has no nodes between $x_1$ and $x_2$, then it must have the same sign everywhere 
    in the interval. Show that (b) then leads to a contradiction. Therefore, between every pair of nodes of
    $\psi_m(x), ~ \psi_n(x)$ must have at \emph{least} one node, and in particular the number of nodes 
    increases with energy.

    \textcolor{hwColor}{
      The goal is to show that the number of nodes in any stationary state increases with increasing energy. Suppose we have 
      two real linearly independent solutions $\psi_n(x)$ and $\psi_m(x)$ with corresponding energies $E_n$ and $E_m$ where $E_n > E_m$. \\ \\ 
      The Schr$\ddot{o}$dinger equation is $i\hbar \dfrac{\partial \Psi}{\partial t}=-\dfrac{\hbar^2}{2m}\dfrac{\partial^2 \Psi}{\partial x^2}+V(x,t) \Psi(x,t)$ \\
      \\
      \\
      Assume $V(x,t)=V(x)$, then we have 
      $
        \begin{cases}
          E=i \hbar \dfrac{\phi^'(t)}{\phi(t)} ~~~~~~~~~~~~~~~~~~ \mathbf{(A)} \\
          \\
          E=V(x)-\dfrac{\hbar^2}{2m}\dfrac{\psi^{''}(x)}{\psi(x)} ~~~~ \mathbf{(B)}
        \end{cases} \\ \\
      $
      We can tell recognize that \textbf{(B)} is s the time-independent Schr$\ddot{o}$dinger equation, therefore we can write; \\
      \\
      \\
      $
        \begin{cases}
          \dfrac{d^2 \psi_m}{dx^2}=\dfrac{2m}{\hbar^2} \psi_m \left(V(x)-E_m\right) \\
          \\
          \dfrac{d^2 \psi_n}{dx^2}=\dfrac{2m}{\hbar^2} \psi_n \left(V(x)-E_n\right) \\
        \end{cases}
      $ 
      \\
      \\
      By doing some algebra we can get the needed/asked result. \\
      \\
      \\
      $
        \psi_n \dfrac{d^2 \psi_m}{dx^2}-\psi_m \dfrac{d^2 \psi_n}{dx^2}=\dfrac{2m}{\hbar^2}\left[\left(V(x)-E_m\right)\psi_m \psi_n- \left(V(x)-E_n\right)\psi_m \psi_n \right] \\
        \\
        =\dfrac{2m}{\hbar^2}\psi_m \psi_n \left(E_n-E_m\right) \\
        \\
        \\
        \\
        \psi_n \dfrac{d^2 \psi_m}{dx^2}-\psi_m \dfrac{d^2 \psi_n}{dx^2}+\dfrac{d \psi_n}{dx}\dfrac{d \psi_m}{dx}-\dfrac{d \psi_n}{dx}\dfrac{d \psi_m}{dx}=\dfrac{2m}{\hbar^2}\psi_m \psi_n \left(E_n-E_m\right) \\
        \\
        \\
        \\
        \dfrac{d}{dx}\left(\psi_n \dfrac{d \psi_m}{dx}-\psi_m\dfrac{d\psi_n}{dx}\right)=\dfrac{2m}{\hbar^2}\psi_m \psi_n \left(E_n-E_m\right) \\
        \\
        \\
        \rule{15cm}{1pt}
        \\
        \\
        \bigints_{x_1}^{x_2} \dfrac{d}{dx}\left(\psi_n \dfrac{d \psi_m}{dx}-\psi_m\dfrac{d\psi_n}{dx}\right) dx=\bigints_{x_1}^{x_2} \dfrac{2m}{\hbar^2}\psi_m \psi_n \left(E_n-E_m\right) dx=\dfrac{2m}{\hbar^2} \left(E_n-E_m\right) \bigints_{x_1}^{x_2} \psi_m \psi_n dx \\
        \\
        \\
        \left[x_2  \psi_n(x_2) \dfrac{\psi_m}{dx}-x_2 \psi_m(x_2)\dfrac{d \psi_n}{dx}-x_1 \psi_n(x_1) \dfrac{d \psi_m}{dx}+x_1 \psi_m(x_1)\dfrac{d \psi_n}{dx}\right]=\dfrac{2m}{\hbar^2} \left(E_n-E_m\right) \bigints_{x_1}^{x_2} \psi_m \psi_n dx \\ 
        \\
      $
      \\
      Keep in mind that nodes are where a function is zero. \\
      \\
      \\
      $
        \therefore ~~~ \psi^'_m(x_2)\psi_n(x_2)-\psi^'_m(x_1)\psi_n(x_1)=\dfrac{2m}{\hbar^2} \left(E_n-E_m\right) \bigints_{x_1}^{x_2} \psi_m \psi_n dx \\
        \\
      $
      \textbf{Conclusion:} \\
      \\
      Suppose $\psi_n(x)$ does not have any nodes between $x_1$ to $x_2$, then $\psi_n(x)$ has the same sign in that interval. 
      Because $\psi_m(x)$ does not have any nodes, hence, $\psi_m$ has the same sign in the interval as well. \\
      Based on the calculated equation, $\psi_n$ has at least one node between $x_1$ and $x_2$. We can Conclude as the number of 
      nodes increases we have increase in enegry as well. 
    }
  
  \end{itemize}

  \textbf{2.46}
  \\
  Imagine a bead of mass $m$ that slides frictionlessly around a circular wire ring of circumference $L$. 
  (This is just like a free particle, except that $\psi(x+L)=\psi(x)$) Find the stationary states (with
  appropriate normalization) and the corresponding allowed energies. Note that there are (with one exception)
  \emph{two} independent solutions for each energy $E_n$ corresponding to clockwise and counterf-clockwise
  circulation; call them $\psi^+_n(x)$ and $\psi^-_n(x)$. How do you account for this degeneracy, in
  view of the theorem in Problem 2.44 (why does the theorem fail, in this case)?

  \textcolor{hwColor}{
    \\
    Like the previous problems, we need to start with the Schr$\ddot{o}$dinger equation is: \\
    \\
    $
      i\hbar \dfrac{\partial \Psi}{\partial t}=-\dfrac{\hbar^2}{2m}\dfrac{\partial^2 \Psi}{\partial x^2}+V(x,t) \Psi(x,t)
    $
    , where $-\infty<x<+\infty$ and $t>0$. \\
    \\
    \\
    \textbf{A:} We need to solve the Schr$\ddot{o}$dinger equation. \\ 
    \\
    \textbf{B:} The bead is assumed to be frictionless, hence, $V(x,t)=0$. \\
    \\
    \\
    $
      i\hbar \dfrac{\partial \Psi}{\partial t}=-\dfrac{\hbar^2}{2m}\dfrac{\partial^2 \Psi}{\partial x^2} \\ \\
    $
    As this problem states $\Psi(x,t)=\Psi(x+L,t) \Longrightarrow \dfrac{\partial \Psi}{\partial x}(x,t)=\dfrac{\partial \Psi}{\partial x}(x+l, t)$. \\ \\
    \emph{"These periodic boundary conditions are different from the usual Dirichlet boundary conditions
    $\Psi \rightarrow 0 as x \rightarrow \pm \infty$ and are the reason why the degeneracy theorem in Problem 2.44 fails.
    "} \\ \\
    After some gooling, I learned that for simplicity's sake, we should set $x=-\dfrac{L}{2}$ (For symmetry and simplify). \\
    \\
    $$
      \dfrac{\partial \Psi}{\partial x}(-\dfrac{L}{2},t)=\dfrac{\partial \Psi}{\partial x}(+\dfrac{L}{2}, t)
    $$
    \\
    \\
    Our goal is to find the stationary states and their corresponding energies. We can use the method of separation of variables.
    Assume a solution is $\Psi(x,t)=\psi(x) \phi(t)$, thus: \\
    \\
    $
      i\hbar \dfrac{\partial}{\partial t} \left(\psi(x) \phi(t)\right)=-\dfrac{\hbar^2}{2m} \dfrac{\partial^2}{\partial x^2} \left(\psi(x) \phi(t)\right) \\
      \\
      \\
      \begin{cases}
        \dfrac{\partial \Psi}{\partial x}(-\dfrac{L}{2},t)=\dfrac{\partial \Psi}{\partial x}(\dfrac{L}{2}) \\
        \\
        \Psi(-\dfrac{L}{2},t)=\Psi(\dfrac{L}{2},t)
      \end{cases} \Longrightarrow \begin{cases}
        \psi^'(-\dfrac{L}{2}) \phi(t)=\psi^'(\dfrac{L}{2})\phi(t) \\
        \\
        \psi(-\dfrac{L}{2}) \phi(t)=\psi(\dfrac{L}{2})\phi(t)
      \end{cases} \Longrightarrow \begin{cases}
        \psi^'(-\dfrac{L}{2})=\psi^'(\dfrac{L}{2}) \\
        \\
        \psi(-\dfrac{L}{2}) =\psi(\dfrac{L}{2})
      \end{cases}
    $ \\
    \\
    Let's get back to our earlier calculated Schr$\ddot{o}$dinger equation. Divide its both sides by 
    $\psi(x) \phi(t)$ (assumed solution). Then what we end up getting is a function of $t$ on one side and a function of $x$
    on the other side. \\ \\
    Mathematically, speaking the two are \emph{only} equal \emph{when} both sides are equal to a constant. \\
    \\
    \\
    $
      \hbar i\dfrac{\dfrac{d \phi(t)}{dt}}{\phi(t)}=-\dfrac{\hbar^2}{2m} \dfrac{\dfrac{d^2 \psi(x)}{dx^2}}{\psi(x)}=E \Longrightarrow \begin{cases}
        E=\hbar i \dfrac{\dfrac{d \phi(t)}{dt}}{\phi(t)} \\
        \\
        E=-\dfrac{\hbar^2}{2m} \dfrac{\dfrac{d^2 \psi(x)}{dx^2}}{\psi(x)}
      \end{cases}
    $ \\
    \\
    "Values of $E$ for which the boundary conditions are satisfied are called the eigenvalues (or
    eigenenergies in this context), and the nontrivial solutions associated with them are called the
    eigenfunctions (or eigenstates in this context). The ODE in x is known as the time-independent
    Schr$\ddot{o}$dinger equation (TISE) and can be written as:" ~ Dr.Google $\ddot\smile$ \\
    \\
    \\
    \textbf{1) If there are negative eigenvalues: $E=-\gamma^2$} \\ \\
    $
      \dfrac{d^2 \psi}{dx^2}=-\dfrac{2mE \psi}{\hbar^2}=\dfrac{2mE \gamma^2}{\hbar^2}, ~~~~ -\dfrac{L}{2}<x<\dfrac{L}{2} \\
      \\
      \psi(x)=C_1 cosh \left(\dfrac{\sqrt{2m}\gamma x}{\hbar}\right)+C_2 sinh \left(\dfrac{\sqrt{2m}\gamma x}{\hbar}\right)
    $
    \\
    \\
    Time to apply the conditions we found earlier: \\ \\
    $
      \begin{cases}
        \psi^'(-\dfrac{L}{2})=\psi^'(\dfrac{L}{2}) \\
        \\
        \psi(-\dfrac{L}{2}) =\psi(\dfrac{L}{2})
      \end{cases} \Longrightarrow \begin{cases}
        2 \dfrac{\sqrt{2m} C_1 \gamma}{\hbar} sinh \left(\dfrac{\sqrt{2m} \gamma L}{2\hbar}\right)=0
        \\
        2C_2 sinh \left(\dfrac{\sqrt{2m} \gamma L}{2\hbar}\right)=0
      \end{cases} \Longrightarrow \begin{cases}
        C_1=0 \\ 
        C_2=0
      \end{cases} \\ \\
    $
    \rule{15cm}{1pt}
    \\ 
    \\
    \textbf{2) If zero is an eigenvalue: $E=0$} \\ \\
    $
      \dfrac{d^2 \psi}{dx^2}=0 \Rightarrow \psi(x)=C_3x+C_4 \\
      \\
      \begin{cases}
        \psi^'(-\dfrac{L}{2})=\psi^'(\dfrac{L}{2}) \\
        \\
        \psi(-\dfrac{L}{2}) =\psi(\dfrac{L}{2})
      \end{cases} \Longrightarrow \begin{cases}
        C_4-\dfrac{C_3 L}{2}=C_4+\dfrac{C_3 L}{2} \\ 
        C_3=C_3
      \end{cases} \\ \\
      \\
      \\
      \\
      \bigint_{-\dfrac{L}{2}}^{+\dfrac{L}{2}} \psi^2(x) dx=1 \Rightarrow C_4=\dfrac{1}{\sqrt{L}} \\
      \\
      \\
    $
    \rule{15cm}{1pt}
    \\ 
    \\
    \textbf{3) If there are positive eigenvalues: $E=\mu^2$} \\ \\
    $
      \dfrac{d^2 \psi}{dx^2}=-\dfrac{2m \mu^2 \psi}{\hbar^2} \Rightarrow \psi(x)=C_5 cos \left(\dfrac{\sqrt{2m} \mu x}{\hbar}\right)+C_6 sin \left(\dfrac{\sqrt{2m}\mu x}{\hbar}\right) \\
      \\
      \\
      \begin{cases}
        \psi \left(-\dfrac{L}{2}\right)=\psi \left(\dfrac{L}{2}\right) \\
        \\
        \psi^' \left(-\dfrac{L}{2}\right)=\psi^' \left(\dfrac{L}{2}\right)
      \end{cases} \Longrightarrow \begin{cases}
        2C_6 sin\dfrac{\sqrt{2m} \mu L}{2\hbar}=0 \\
        \\
        2 \dfrac{\sqrt{2m} C_5 \mu}{\hbar} sin \dfrac{\sqrt{2m} \mu L}{2\hbar}=0
      \end{cases} \\ 
    $
    \\
    When $sin \dfrac{\sqrt{2m} \mu L}{2\hbar}=0$ then, $\mu=\sqrt{\dfrac{2}{m}} \dfrac{n \pi \hbar}{L}$ where $n=1, 2, 3,...$. Hence: \\
    \\
    $
      E_n=\dfrac{2n^2 \pi^2 \hbar^2}{mL^2} \Longrightarrow \psi(x)=C_5 cos \dfrac{2n \pi x}{L}+C_6 sin \dfrac{2n \pi x}{L} \\ \\
      \\
      \begin{cases}
        \bigint_{-\dfrac{L}{2}}^{+\dfrac{L}{2}} \left(C_5 cos\dfrac{2n \pi x}{L}\right)^2 dx=1 \\
        \\
        \bigint_{-\dfrac{L}{2}}^{+\dfrac{L}{2}} \left(C_6 sin \dfrac{2n \pi x}{L}\right)^2 dx=1 \\
      \end{cases} \Longrightarrow \begin{cases}
        C_5=\sqrt{\dfrac{2}{L}} \\
        \\
        C_6=\sqrt{\dfrac{2}{L}}
      \end{cases}
    $ \\
    \\
    For $\phi(t)=e^{-i E_n t\hbar}$. The stationary states are: \\
    \\
    \\
    $
      \begin{cases}
        \Psi_0(x,t)= \psi_0(x) \phi_0(t)=\dfrac{1}{\sqrt{L}} \\
        \\
        \Psi_n(x,t)=\sqrt{\dfrac{2}{L}} cos \dfrac{2n \pi x}{L} exp\{-i E_n \dfrac{t}{\hbar}\}=\sqrt{\dfrac{2}{L}} cos \dfrac{2n \pi x}{L} exp \{ -i\dfrac{2n^2 \pi^2 \hbar t}{mL^2}\} \\
        \\
        \Psi_n(x,t)=\sqrt{\dfrac{2}{L}} sin \dfrac{2n \pi x}{L} exp\{-i E_n \dfrac{t}{\hbar}\}=\sqrt{\dfrac{2}{L}} sin \dfrac{2n \pi x}{L} exp \{ -i\dfrac{2n^2 \pi^2 \hbar t}{mL^2}\}
      \end{cases} \\ \\
    $
    \\
    Hence, the allowed energies are $E=0$ and $E_n=\dfrac{2 n^2 \pi^2 \hbar^2}{mL^2}$. \emph{"According to the
    principle of superposition, the general solution to the Schr¨odinger equation is a linear
    combination of these stationary states"}. \\
    \\
    $
      \Psi(x,t)=A_0 \dfrac{1}{\sqrt{L}}+\sum\limits_{n=1}^{\infty}A_n \sqrt{\dfrac{2}{L}} exp \{ -i\dfrac{2n^2 \pi^2 \hbar t}{mL^2}\} cos \dfrac{2n\pi x}{L}+\sum\limits_{n=1}^{\infty}B_n \sqrt{\dfrac{2}{L}} exp \{ -i\dfrac{2n^2 \pi^2 \hbar t}{mL^2}\} sin \dfrac{2n\pi x}{L}
    $
  }

\end{document}
