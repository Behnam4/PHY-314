\documentclass[fleqn]{article}
\oddsidemargin 0.0in
\textwidth 6.0in
\thispagestyle{empty}
\usepackage{import}
\usepackage{amsmath}
\usepackage{graphicx}
\usepackage{flexisym}
\usepackage{amssymb}
\usepackage{bigints} 
\usepackage[english]{babel}
\usepackage[utf8x]{inputenc}
\usepackage{float}
\usepackage[colorinlistoftodos]{todonotes}

\definecolor{hwColor}{HTML}{AD53BA}

\begin{document}

  \begin{titlepage}

    \newcommand{\HRule}{\rule{\linewidth}{0.5mm}}

    \center


    \textsc{\LARGE Arizona State University}\\[1.5cm]

    \textsc{\LARGE Quantum Physics I }\\[1.5cm]


    \begin{figure}
      \includegraphics[width=\linewidth]{asu.png}
    \end{figure}


    \HRule \\[0.4cm]
    { \huge \bfseries Homework Seven}\\[0.4cm] 
    \HRule \\[1.5cm]

    \textbf{Behnam Amiri}

    \bigbreak

    \textbf{Prof: Richard Kirian}

    \bigbreak


    \textbf{{\large \today}\\[2cm]}

    \vfill 

  \end{titlepage}

  \textbf{1.14} \\ \\
  Let $P_{ab}(t)$ be the probability of finding the particle in the range $(a<x<b)$, at time $t$.
  \begin{itemize}
    \item Show that $$\dfrac{dP_{ab}}{dt}=J(a,t)-J(b,t)$$ \\
    , where \\
    $$J(x,t)\equiv \dfrac{i \hbar}{2m} \left(\Psi \dfrac{\partial \Psi^*}{\partial x}-\Psi^* \dfrac{\partial \Psi}{\partial x}\right)$$
    What are the units of $J(x,t)?$ Comment: $J$ is called the \textbf{probability current}, because it tells you the rate at which 
    probability is "flowing" past the point $x$. If $P_{ab}(t)$ is increasing, then more probability is flowing into the region
    at one end than flows out at the other.


      \textcolor{hwColor}{
        In the simplest form,Born’s interpretation states that the probability density of finding a particle at a given point, 
        when measured, is proportional to the square of the magnitude of the particle's wavefunction at that point. \\
        \\
        $$
          P_{ab}(t)= \bigints_{a}^{b} |\Psi(x,t)|^2 dx=\bigints_{a}^{b} \Psi(x,t) \Psi^*(x,t) dx \\ \\
        $$
        $
          \dfrac{dP_{ab}}{dt}=\dfrac{d}{dt} \left(\bigints_{a}^{b} \Psi(x,t) \Psi^*(x,t) dx\right)=\bigints_{a}^{b} \dfrac{\partial}{\partial t} \left(\Psi(x,t) \Psi^*(x,t)\right) dx \\
          \\
          =\bigints_{a}^{b} \left[\dfrac{\partial \Psi}{\partial t}\Psi^*+\dfrac{\partial \Psi^*}{\partial t}\Psi\right] dx \\ \\
        $
        From the textbook we have the Schr$\ddot{o}$dinger equation as: \\
        \\
        $
          i\hbar \dfrac{\partial \Psi}{\partial t}=-\dfrac{\hbar^2}{2m}\dfrac{\partial^2 \Psi}{\partial x^2}+V \Psi
          \Rightarrow \dfrac{\partial \Psi}{\partial t}=-\dfrac{\hbar}{2mi}\dfrac{\partial^2 \Psi}{\partial x^2}+\dfrac{1}{i\hbar}V \Psi \\
        $
        \\
        \\
        Let's multiple both sides by $\dfrac{i}{i}$, then we get: \\
        \\
        $
          \dfrac{\partial \Psi}{\partial t}=\dfrac{\hbar i}{2m} \dfrac{\partial^2 \Psi}{\partial x^2}-\dfrac{i}{\hbar} V \Psi 
          \Rightarrow \dfrac{\partial \Psi^*}{\partial t}=-\dfrac{\hbar i}{2m} \dfrac{\partial^2 \Psi^*}{\partial x^2}+\dfrac{i}{\hbar} V \Psi^* \\ \\
          \\
        $
        Now, we are ready to evaluate $ \dfrac{dP_{ab}}{dt}$; \\
        \\
        $
          \dfrac{dP_{ab}}{dt}=\bigints_{a}^{b} \left[\left(\dfrac{\hbar i}{2m} \dfrac{\partial^2 \Psi}{\partial x^2}-\dfrac{i}{\hbar} V \Psi\right) \Psi^*+\left(-\dfrac{\hbar i}{2m} \dfrac{\partial^2 \Psi^*}{\partial x^2}+\dfrac{i}{\hbar} V \Psi^*\right)\Psi\right] dx=\bigints_{a}^{b} \dfrac{\hbar i}{2m} \left[\dfrac{\partial^2 \psi}{\partial x^2}\Psi^*-\dfrac{\partial \Psi^*}{\partial x^2}\Psi\right]dx \\
          \\
          \\
          =\dfrac{\hbar i}{2m} \bigints_{a}^{b} \left[\left(\dfrac{\partial \Psi^*}{\partial x}\dfrac{\partial \Psi}{\partial x}+\dfrac{\partial^2 \Psi}{\partial x^2} \Psi^*\right)-\left(\Psi \dfrac{\partial^2 \Psi^*}{\partial x^2}+\dfrac{\partial \Psi^*}{\partial x} \dfrac{\partial \Psi}{\partial x}\right)\right]dx \\
          \\
          \\
          =\dfrac{\hbar i}{2m} \bigints_{a}^{b} \dfrac{\partial}{\partial x} \left[\dfrac{\partial \Psi}{\partial x} \Psi^* -\Psi \dfrac{\partial \Psi^*}{\partial x}\right]dx \\
          \\
          \\
          =\dfrac{\hbar i}{2m} \left[\left(\Psi^*(b,t) \dfrac{\partial \Psi}{\partial x}(b,t)-\Psi(b,t) \dfrac{\Psi^*}{\partial x}(b,t)\right)-\left(\Psi^*(a,t) \dfrac{\partial \Psi}{\partial x}(a,t)-\Psi(a,t) \dfrac{\Psi^*}{\partial x}(a,t)\right)\right] \\
          \\
          \\
          \\
          \therefore ~~~ \dfrac{dP_{ab}}{dt}=\mathcal{J}(b,t)-\mathcal{J}(a,t) \\ \\
        $
        $P_{ab}$ is dimensionless, therefore $\mathcal{J}$ has $\dfrac{1}{s}$ as its unit. \\
        \\
        \\
        $
          J(x,t)=\dfrac{i \hbar}{2m} \left(\Psi \dfrac{\partial \Psi^*}{\partial x}-\Psi^* \dfrac{\partial \Psi}{\partial x}\right) \\
          \\
          \\
          =\dfrac{i \hbar}{2m}\left(\sqrt[4]{\dfrac{2am}{\pi \hbar}}e^{-a\left[(mx^2/\hbar)+it\right]}\right) \dfrac{\partial}{\partial x} \left(\sqrt[4]{\dfrac{2am}{\pi \hbar}}e^{-a\left[(mx^2/\hbar)-it\right]}\right) \\ \\
          -\dfrac{i \hbar}{2m}\left(\sqrt[4]{\dfrac{2am}{\pi \hbar}}e^{-a\left[(mx^2/\hbar)+it\right]}\right) \dfrac{\partial}{\partial x} \left(\sqrt[4]{\dfrac{2am}{\pi \hbar}}e^{-a\left[(mx^2/\hbar)-it\right]}\right)
          \\
          \\
          \\
          \therefore ~~~  J(x,t)=0
        $
      }

    \item Find the probability current for the wave function in Problem 1.9. (This is not a very pithy example, I am afraid
    ; we will encounter more substantial ones in due course.)
  \end{itemize}

  \rule{15cm}{1pt}

  \textbf{2.18}
  \begin{itemize}
    \item Find the probability current, J (Problem 1.14) for the free particlen wave function Equation 2.95. which
    direction does the probability flow?
  \end{itemize}

  \rule{15cm}{1pt}

  \textbf{2.20} \\ \\
  A free particle has the initial wave function \\
  $$\Psi(x,0)=Ae^{-a|x|}$$
  where $A$ and $a$ are positive real constants.
  \begin{itemize}
    \item Normalize $\Psi(x,0)$.

    \item Find $\phi(k)$.

    \item Construct $\Psi(x,t)$, in the form of an integral.

    \item Discuss the limiting cases (a very large, and a very small).
  \end{itemize}

  \rule{15cm}{1pt}

  \textbf{2.45} \\ \\
  In this problem you will show that the number of nodes of the stationary states of a one-dimensional potential always 
  increases with energy. Consider two (real, normalized) solutions ($\psi_n$ and $\psi_m$) to the time-independent 
  Schr$\ddot{o}$dinger equation (for a given potential $V(x)$), with energies $E_n > E_m$.
  \begin{itemize}
    \item Show that \\
    $$\dfrac{d}{dx} \left(\dfrac{d \psi_m}{dx}\psi_n-\psi_m\dfrac{d \psi_n}{dx}\right)=\dfrac{2m}{\hbar^2}(E_n-E_m) \psi_m \psi_n$$

    \item Let $x_1$ and $x_2$ be two adjacent nodes of the function $\psi_m(x)$.Show that \\
    $$\psi^'_m(x_2) \psi_n(x_2)-\psi^'_m(x_1) \psi_n(x_1)=\dfrac{2m}{\hbar^2}(E_n-E_m) \bigints_{x_1}^{x_2} \psi_m \psi_n dx$$

    \item If $\psi_n(x)$ has no nodes between $x_1$ and $x_2$, then it must have the same sign everywhere 
    in the interval. Show that (b) then leads to a contradiction. Therefore, between every pair of nodes of
    $\psi_m(x), ~ \psi_n(x)$ must have at \emph{least} one node, and in particular the number of nodes 
    increases with energy.
  \end{itemize}

  \rule{15cm}{1pt}

  \textbf{2.46}
  \\
  Imagine a bead of mass $m$ that slides frictionlessly around a circular wire ring of circumference $L$. 
  (This is just like a free particle, except that $\psi(x+L)=\psi(x)$) Find trhe stationary states (with
  appropriate normalization) and the corresponding allowed energies. Note that there are (with one exception)
  \emph{two} independent solutions for each energy $E_n$ corresponding to clockwise and counterf-clockwise
  circulation; call them $\psi^+_n(x)$ and $\psi^-_n(x)$. How do you account for this degeneracy, in
  view of the theorem in Problem 2.44 (why does the theorem fail, in this case)?

\end{document}
