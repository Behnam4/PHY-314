\documentclass[fleqn]{article}
\oddsidemargin 0.0in
\textwidth 6.0in
\thispagestyle{empty}
\usepackage{import}
\usepackage{amsmath}
\usepackage{graphicx}
\usepackage{flexisym}
\usepackage{amssymb}
\usepackage{bigints} 
\usepackage[english]{babel}
\usepackage[utf8x]{inputenc}
\usepackage{float}
\usepackage[colorinlistoftodos]{todonotes}

\definecolor{hwColor}{HTML}{AD53BA}

\begin{document}

  \begin{titlepage}

    \newcommand{\HRule}{\rule{\linewidth}{0.5mm}}

    \center


    \textsc{\LARGE Arizona State University}\\[1.5cm]

    \textsc{\LARGE Quantum Physics I }\\[1.5cm]


    \begin{figure}
      \includegraphics[width=\linewidth]{asu.png}
    \end{figure}


    \HRule \\[0.4cm]
    { \huge \bfseries Homework Ten}\\[0.4cm] 
    \HRule \\[1.5cm]

    \textbf{Behnam Amiri}

    \bigbreak

    \textbf{Prof: Richard Kirian}

    \bigbreak


    \textbf{{\large \today}\\[2cm]}

    \vfill 

  \end{titlepage}


  \textbf{2.13 Zettili} \\ \\
  In the following expressions, where $\hat{A}$ is an operator, specify the nature of each expression 
  (i.e., specify whether it is an operator, a bra, or a ket); then find its Hermitian conjugate.
  \begin{itemize}
    \item $\langle \phi | \hat{A} | \psi \rangle \langle  \psi |$

    \item $\hat{A} | \psi \rangle \langle \phi |$
    
    \item $\langle \phi | \hat{A} | \psi \rangle | \psi \rangle \langle \phi | \hat{A}$
    
    \item $\langle \psi | \hat{A} | \phi \rangle | \phi \rangle + i \hat{A} | \psi \rangle$
    
    \item $\left(| \phi \rangle  \langle \phi | \hat{A} \right)-i\left(\hat{A} | \psi \rangle \langle \psi |\right)$
  \end{itemize}

  \rule{15cm}{1pt}

  \textbf{3.5} \\ \\
  \begin{enumerate}
    \item Find the Hermitian conjugates of $x, i$ and $\dfrac{d}{dx}$

      \textcolor{hwColor}{
        Let's define the Hermitian conjugates before we start. The Hermitian conjugates 
        of an operator $\hat{S}$ is $\hat{S}^{\dagger}$ so that 
        $$\langle f | \hat{S} g \rangle= \langle\hat{S}^{\dagger} f | g \rangle$$
        \\
        $
          \langle f | xg \rangle=\bigints\limits_{-\infty}^{+\infty} xgf^* dx=\bigints\limits_{-\infty}^{+\infty} g(xf)^* dx=\langle x f|g \rangle \\
          \\
          \\
          \therefore ~~~ x=x^{\dagger} ~~~ \surd 
          \\
          \\
          \rule{15cm}{1pt}
          \\
          \\
          \langle f | ig \rangle=\bigints\limits_{-\infty}^{+\infty} igf^* dx=\bigints\limits_{-\infty}^{+\infty} g(-if)^* dx=\langle -i f|g \rangle \\
          \\
          \\
          \therefore ~~~ i^{\dagger}=-i ~~~ \surd
          \\
          \\
          \rule{15cm}{1pt}
          \\
          \\
          \langle f | \dfrac{d g}{dx} \rangle=\bigints\limits_{-\infty}^{+\infty} \dfrac{dg}{dx} f^* dx=-\bigints\limits_{-\infty}^{+\infty} g\left(\dfrac{df}{dx}\right)^* dx+f^* g\Big|_{-\infty}^{+\infty}=-\bigints\limits_{-\infty}^{+\infty} g\left(\dfrac{df}{dx}\right)^* dx \\
          \\
          \\
          \therefore ~~~ \left(\dfrac{d}{dx}\right)^{\dagger}=-\dfrac{d}{dx} ~~~ \surd
        $
      }

    \item Show that $(\hat{Q} \hat{R})^{\dagger}=\hat{R}^{\dagger} \hat{Q}^{\dagger}$ 
    (note the reversed order), $(\hat{Q}+\hat{R})=\hat{Q}^{\dagger}+\hat{R}^{\dagger}$, and 
    $(c \hat{Q})^{\dagger}=c^* \hat{Q}^{\dagger}$ for a complex number c.

      \textcolor{hwColor}{
        $
          \langle f| \hat{Q} \hat{R} g \rangle=\langle \hat{Q}^{\dagger} f | Rg \rangle=\langle \hat{R}^{\dagger} \hat{Q}^{\dagger} f | g \rangle \\
          \\
          \\
          Note: ~~~ \langle f |\hat{Q} \hat{R} g \rangle=\langle \left(\hat{Q} \hat{R}\right)^{\dagger} f | g \rangle \\
          \\
          \\
          \therefore ~~~  \langle \hat{R}^{\dagger} \hat{Q}^{\dagger} f | g \rangle=\langle \left(\hat{Q} \hat{R}\right)^{\dagger} f | g \rangle 
          \Longrightarrow \left(\hat{Q} \hat{R}\right)^{\dagger}= \hat{R}^{\dagger} \hat{Q}^{\dagger} ~~~ \surd
        $
      }

    \item Construct the Hermitian conjugate of $a_+$ (Equation 2.48).

      \textcolor{hwColor}{
        $
          a_+=\dfrac{1}{\sqrt{2 \hbar m \omega}} \left(m \omega x-ip\right)
          =\dfrac{1}{\sqrt{2 \hbar m \omega}} \left[m \omega x-i\left(-i\hbar \dfrac{\partial}{\partial x}\right)\right]
          =\dfrac{1}{\sqrt{2 \hbar m \omega}} \left(m \omega x-\hbar \dfrac{\partial}{\partial x} \right) \\
          \\
          \\
          \langle f | a_+ g \rangle=\dfrac{1}{\sqrt{2 \hbar m \omega}} \left( m \omega \bigints\limits_{-\infty}^{+\infty} xgf^* dx-\bigints\limits_{-\infty}^{+\infty} \hbar f^* \left(\dfrac{\partial g}{\partial x}\right) dx\right) \\
          \\
          =\dfrac{1}{\sqrt{2 \hbar m \omega}}  \left[
            m \omega \bigints\limits_{-\infty}^{+\infty} (xf)^* g dx+ \bigints\limits_{-\infty}^{+\infty} \hbar \left(\dfrac{\partial f}{\partial x}\right)^* gdx
          \right] 
          \\
          \\
          =\langle a^{\dagger}_+ f | g \rangle=\langle a_- f|g \rangle \\
          \\
          \\
          \therefore ~~~ a^{\dagger}_+=a_- ~~~ \surd  
        $
      }

  \end{enumerate}

  \rule{15cm}{1pt}

  \textbf{3.7} \\ \\
  \begin{itemize}
    \item Suppose that $f(x)$ and $g(x)$ are two eigenfunctions of an operator $\hat{Q}$,
    with the same eigenvalue $q$. Show that any linear combination of $f$ and $q$ is
    itself an eigenfunction of $\hat{Q}$, with eigenvalue $q$.

      \textcolor{hwColor}{
        $
          \begin{cases}
            \hat{Q} f=qf \\
            \\
            \hat{Q} g=qg
          \end{cases} 
          \\
          \\
          h(x)=af(x)+bg(x)
        $ \\
        \\
        $a$ and $b$ are constants. \\
        \\
        $
          \hat{Q} h= \hat{Q} \left(bg+af\right)=a \left(\hat{Q} f\right)+b \left(\hat{Q} g\right)=a(qf)+b(qg) 
          \Longrightarrow \hat{Q} h=qh 
        $ \\
        \\
        We just showed that the linear combination of $f$ and $q$ is an eigenfunction. \\
      }


    \item Check that $f(x)=exp(x)$ and $g(x)=exp(-x)$ are eigenfunctions of
    the operator $\dfrac{d^2}{dx^2}$, with the same eigenvalue. Construct two linear
    combinations of $f$ and $g$ that are orthogonal eigenfunctions on the interval $(-1,1)$.

      \textcolor{hwColor}{
        $
          \begin{cases}
            \dfrac{d^2 f}{dx^2}=\dfrac{d^2}{dx^2} \left(e^x\right)=e^x=f \\
            \\
            \dfrac{d^2 g}{dx^2}=\dfrac{d^2}{dx^2} \left(e^{-x}\right)=e^{-x}=g
          \end{cases}
        $ \\
        \\
        \\
        Hence, $f(x)$ and $g(x)$ are eigenfunctions of $\dfrac{d^2}{dx^2}$ witht the same eigenvalue of 1. \\
        \\
        Two linear combinations: \\
        \\
        $
          \begin{cases}
            cosh(x)=\dfrac{e^x+e^{-x}}{2}=\dfrac{f+g}{2} \\
            \\
            sinh(x)=\dfrac{e^x-e^{-x}}{2}=\dfrac{f-g}{2} 
          \end{cases}
        $ \\
        \\
        The two linear combinations are $\bot. ~~~~ \surd$
      }
    
  \end{itemize}


  \rule{15cm}{1pt}

  \textbf{3.10} \\ \\
  Is the ground state of the infinite square well an eigenfunction of
  momentum? If so, what is its momentum? If not, why not? [For further
  discussion, see Problem 3.34.]

    \textcolor{hwColor}{
      From the textbook we have the wave function of the stationary states of the square well
      $$\psi_n(x)=\sqrt{\dfrac{2}{a}} sin\left(\dfrac{n \pi x}{a}\right)$$ 
      $
        \hat{p} \psi_n= -i \hbar \dfrac{d}{dx} \left[\sqrt{\dfrac{2}{a}} sin\left(\dfrac{n \pi x}{a}\right)\right] \\
        \\
        =-i\hbar \sqrt{\dfrac{2}{a}} \dfrac{n \pi}{a} cos\left(\dfrac{n \pi x}{a}\right) \\
        \\
        \\
        \langle p \rangle=\bigints\limits_{0}^{a} \psi_n \psi^*_n \hat{p} dx
        =-i \dfrac{2\hbar n \pi}{a^2} \bigints\limits_{0}^{a} sin\left(\dfrac{n \pi x}{a}\right) cos\left(\dfrac{n \pi x}{a}\right) \\
        \\
        \\
        \therefore ~~~ \langle p \rangle=0 ~~~ \surd \\
        \\
      $
      Therefore, $\psi_n$ is not an eigenfunction of momentum.
    }


  \rule{15cm}{1pt}

  \textbf{3.33} \\ \\
  An operator $\hat{A}$, representing observable
  A, has two (normalized) eigenstates $\psi_1$ and $\psi_2$ with eigenvalues $a_1$ and $a_2$
  ,respectively. Operator $\hat{B}$ representing observable B, has two (normalized)
  eigenstates $\phi_1$ and $\phi_2$, with eigenvalues $b_1$ and $b_2$. The eigenstates are related
  by
  $$\psi_1=\dfrac{3\phi_1+4\phi_2}{5}, ~~~ \psi_2=\dfrac{4\phi_1-3\phi_2}{5}$$
  \begin{itemize}
    \item Observable A is measured, and the value $a_1$ is obtained. What is the state
    of the system (immediately) after this measurement?

      \textcolor{hwColor}{
        After this measurement, the system is in only one of the two $\psi_1$ and $\psi_2$. 
        Since the value of $a_1$ is obtained from observable A then the state of the system 
        (immediately) after this measurement is $\psi_1$.
      }

    \item If B is now measured, what are the possible results, and what are their
    probabilities?

      \textcolor{hwColor}{
        We just showed that the system is in state $\psi_1$ and we are interested in observable B. So let's find
        use $$\psi_1=\dfrac{3\phi_1+4\phi_2}{5}$$
        Based on this equation, there are only two possible outcomes.
        \\
        \begin{itemize}
          \item A) $\phi_1$ and eigenvalue $b_1$
          \\
          \item B) $\phi_2$ and eigenvalue $b_2$ 
        \end{itemize}
      }

      \textcolor{hwColor}{
        We know the probabilities, because we interpret the squared modules of the coefficients in the 
        basis expansion as the probabilities of measuring the certain eigenvalue, therefore the probability
        of $b_1$ is $\left(\dfrac{3}{5}\right)^2=\dfrac{9}{25}$ and the probability of $b_2$ is 
        $\left(\dfrac{4}{5}\right)^2=\dfrac{16}{25}$.
      }

    \item Right after the measurement of B, A is measured again. What is the
    probability of getting $a_1$? (Note that the answer would be quite different if
    I had told you the outcome of the B measurement.)

      \textcolor{hwColor}{
        In this case we do not know the outcome of the B measurement so we cannot know in which state 
        the system is in afterward. what we can do is that sum up the probabilities.
        \\
        \\
        $
          P=\left(\dfrac{9}{25}\right)^2+\left(\dfrac{16}{25}\right)^2 \\
          \\
          \\
          \therefore ~~~ P=0.5392  ~~~ \surd 
        $
      }
    
  \end{itemize}

\end{document}
