\documentclass[fleqn]{article}
\oddsidemargin 0.0in
\textwidth 6.0in
\thispagestyle{empty}
\usepackage{import}
\usepackage{amsmath}
\usepackage{graphicx}
\usepackage{flexisym}
\usepackage{amssymb}
\usepackage{bigints} 
\usepackage[english]{babel}
\usepackage[utf8x]{inputenc}
\usepackage{float}
\usepackage[colorinlistoftodos]{todonotes}

\definecolor{hwColor}{HTML}{AD53BA}

\begin{document}

  \begin{titlepage}

    \newcommand{\HRule}{\rule{\linewidth}{0.5mm}}

    \center


    \textsc{\LARGE Arizona State University}\\[1.5cm]

    \textsc{\LARGE Quantum Physics I }\\[1.5cm]


    \begin{figure}
      \includegraphics[width=\linewidth]{asu.png}
    \end{figure}


    \HRule \\[0.4cm]
    { \huge \bfseries Homework Six}\\[0.4cm] 
    \HRule \\[1.5cm]

    \textbf{Behnam Amiri}

    \bigbreak

    \textbf{Prof: Richard Kirian}

    \bigbreak

    \textbf{{\large \today}\\[2cm]}

    \vfill 

  \end{titlepage}

  \textbf{2.11}
  \begin{itemize}
    \item Compute $<x>, <p>, <x^2>,$ and $<p^2>$, for the states $\psi_0$ (Equation 2.60) and $\psi_1$ (Equation 2.63), by explicit integration. Comment: In this and other problems involving 
    the harmonic oscillator it simplifies matters if you introduce the variable $\xi_\equiv \sqrt{\dfrac{m \omega}{\hbar}}x$ and the constant 
    $\alpha \equiv (\dfrac{m \omega}{\pi \hbar})^{1/4}$

    \item Check the uncertainty principle for those states.

    \item Compute $<T>$ and $<V>$ for these states. (No new integration allowed!) Is their sum what you would expect?

    \textcolor{hwColor}{
      Let's use the first three eigenstates of the harmonic oscillator: \\ \\
      $
        \begin{cases}
          \psi_0(x)=(\dfrac{m \omega}{\pi \hbar})^{1/4} ~ exp(-\dfrac{m \omega}{2\hbar}x^2) \\
          \\
          \psi_1(x)=(\dfrac{4m^3 \omega^3}{\pi \hbar^3})^{1/4}x ~ exp(-\dfrac{m \omega}{2\hbar}x^2) 
        \end{cases} \\ \\
      $
    }

    \textcolor{hwColor}{
      \textbf{For $\psi_0$:} \\ \\
      $
        <x>=\bigints_{-\infty}^{+\infty} \psi^{*}_0(x) \psi_0(x)xdx=\bigints_{-\infty}^{+\infty}x ~ exp(-\dfrac{m \omega}{2\hbar}x^2) ~ exp(-\dfrac{m \omega}{2\hbar}x^2)(\dfrac{m \omega}{\pi \hbar})^{1/4} dx \\
        \\
        =\left(\dfrac{m \omega}{\pi \hbar}\right)\bigints_{-\infty}^{+\infty} x ~ exp(-\dfrac{m \omega}{\hbar} x^2)dx=0 \\ \\
        \therefore ~ <x>=0 ~~~ \surd \\
        \\
        \\
        <x^2>=\bigints_{-\infty}^{+\infty} x^2 \psi^{*}_0(x) \psi_0(x) dx=\bigints_{-\infty}^{+\infty} x^2(\dfrac{m \omega}{\pi \hbar})^{1/4} (\dfrac{m \omega}{\pi \hbar})^{1/4} ~ exp(-\dfrac{m \omega}{2\hbar}x^2) ~ exp(-\dfrac{m \omega}{2\hbar}x^2)(\dfrac{m \omega}{\pi \hbar})^{1/4} dx \\
        \\
        =(\dfrac{m \omega}{\pi \hbar})^{1/2} \bigints_{-\infty}^{+\infty} x^2 ~ exp(-\dfrac{m \omega}{\hbar}x^2)dx=\dfrac{\hbar}{2m \omega} \\
        \\
        \therefore ~ <x^2>=\dfrac{\hbar}{2m \omega} ~~~ \surd\\
        \\
        \\
        \\
        <p>=\bigints_{-\infty}^{+\infty} \left(-i\hbar \dfrac{\partial}{\partial x}\right)\psi^{*}_0(x) \psi_0(x)dx=-i\hbar \bigints_{-\infty}^{+\infty} \psi^{*}_0(x) \dfrac{\partial \psi_0}{\partial x}dx \\
        \\
        =-i \hbar \bigints_{-\infty}^{+\infty} \left(\dfrac{m \omega}{\pi \hbar}\right)^{1/4} ~ exp(-\dfrac{m \omega}{2\hbar} x^2) \left(\dfrac{m \omega}{\pi \hbar}\right)^{1/4} (-\dfrac{m \omega}{\hbar}x) ~ exp(-\dfrac{m \omega}{2\hbar} x^2) dx \\
        \\
        =m\omega i \sqrt{\dfrac{m \omega}{\pi \hbar}} \bigints_{-\infty}^{+\infty} x ~ exp(-\dfrac{m \omega}{\hbar} x^2) dx=0 \\
        \\
        \therefore ~ <p>=0 ~~~ \surd \\
        \\
        \\
        \\
        \\
        <p^2>=\bigints_{-\infty}^{+\infty} \psi^{*}_0(x) \left(-i\hbar\dfrac{\partial}{\partial x}\right)^2  \psi_0(x)dx=-\hbar^2 \bigints_{-\infty}^{+\infty} \psi^{*}_0(x) \dfrac{d^2 \psi_0(x)}{dx^2}dx \\
        \\
        \\
        =-\hbar^2 \bigints_{-\infty}^{+\infty} \left(\dfrac{m \omega}{\pi \hbar}\right)^{1/4} ~ exp(-\dfrac{m \omega}{2\hbar} x^2) \left(\dfrac{m \omega}{\pi \hbar}\right)^{1/4} \left[exp\left(-\dfrac{m \omega}{2 \hbar}x^2\right) \left(-\dfrac{m \omega}{\hbar}x\right)^2+exp\left(-\dfrac{m \omega}{2\hbar}x^2\right)\left(-\dfrac{m \omega}{\hbar}\right)\right]dx \\
        \\
        \\
        =-\hbar^2 \sqrt{\dfrac{m \omega}{\pi \hbar}} \bigints_{-\infty}^{+\infty} ~ exp\left(-\dfrac{m \omega}{2 \hbar}x^2\right)  \left[\left(-\dfrac{m \omega}{\hbar}x\right)^2+\left(-\dfrac{m \omega}{\hbar}\right)\right]dx \\
        \\
        \\
        =-\hbar^2 \sqrt{\dfrac{m^3 \omega^3}{\pi \hbar^3}} \bigints_{-\infty}^{+\infty} ~ exp\left(-\dfrac{m \omega}{\hbar} x^2\right) \left(\dfrac{m \omega x^2}{\hbar}-1\right)dx \\
        \\
      $
      We, physicists suffer everyday, hence, why not we suffer LESS by introducing a new varaibale $\xi \equiv \sqrt{\dfrac{m \omega}{\hbar}x}$ since it makes our calculation slightly easier. $\ddot\smile$.\\
      \\
      \\
      $
        =-\hbar^2 \sqrt{\dfrac{m^3 \omega^3}{\pi \hbar^3}} \bigints_{-\infty}^{+\infty} \sqrt{\dfrac{\hbar}{m\omega}} \left(\xi^2-1\right) exp(-\xi^2) d\xi=-\dfrac{2\hbar m \omega}{\sqrt{\pi}} \bigints_{0}^{+\infty} \left(\xi^2-1\right) exp(-\xi^2) d\xi \\
        \\
        \\
        =\dfrac{2\hbar m \omega}{\sqrt{\pi}} \left[\bigints_{0}^{+\infty} exp(-\xi^2) d\xi-\bigints_{0}^{+\infty} \xi^2 exp(-\xi^2) d\xi\right]=\dfrac{2\hbar m \omega}{\sqrt{\pi}} \times \dfrac{\sqrt{\pi}}{4} \\
        \\
        \\
        \therefore ~ <p^2>=\dfrac{\hbar m \omega}{2} ~~~ \surd \\
        \\
        \\
        \\
        \begin{cases}
          \sigma_x=\sqrt{<x^2>-<x>^2}=\sqrt{\dfrac{\hbar}{2m \omega}} \\
          \\
          \sigma_p=\sqrt{<p^2>-<p>^2}=\sqrt{\dfrac{\hbar m \omega}{2}}
        \end{cases} \Longrightarrow \sigma_x \sigma_p=\dfrac{\hbar}{2} ~~~ \surd \\ \\
      $
      We just showed that the uncertainty principle $\geq 2$ holds. \\
      \\
      $
        \begin{cases}
          <T>=<\dfrac{p^2}{2m}>=\dfrac{1}{2m}<p^2>=\dfrac{1}{2m}\left(\dfrac{\hbar m \omega}{2}\right)=\dfrac{\hbar \omega}{4} \\
          \\
          <V>=<\dfrac{1}{2}m \omega^2 x^2>=\dfrac{1}{2}m \omega^2<x^2>=\dfrac{1}{2}m \omega^2 \left(\dfrac{\hbar}{2m \omega}\right)=\dfrac{\hbar \omega}{4}
        \end{cases} \\ \\
      $
      Now let's fnd the sum of $<T>$ and $<V>$: \\
      \\
      $
        <T>+<V>=\dfrac{\hbar \omega}{2} \\ \\
      $
      This is what we expected as it is the energy of the ground state. \\
      \\
      \rule{15cm}{1pt}
    }

    \textcolor{hwColor}{
      \textbf{For $\psi_1$:} \\ \\
      $
        <x>=\bigints_{-\infty}^{+\infty} \psi^{*}_1(x) \psi_1(x)xdx=\bigints_{-\infty}^{+\infty} (\dfrac{4m^3 \omega^3}{\pi \hbar^3})^{1/4}x ~ exp(-\dfrac{m \omega}{2\hbar}x^2) (\dfrac{4m^3 \omega^3}{\pi \hbar^3})^{1/4}x ~ exp(-\dfrac{m \omega}{2\hbar}x^2)x dx \\
        \\
        \\
        =\sqrt{\dfrac{4m^3 \omega^3}{\pi \hbar^3}} \bigints_{-\infty}^{+\infty} x^3 exp\left(-\dfrac{m \omega}{\hbar}x^2\right)dx=0 \\
        \\
        \\
        \therefore <x>=0 ~~~ \surd \\
        \\
        \\
        \\
        \\
        <x^2>=\bigints_{-\infty}^{+\infty} \psi^{*}_1(x) \psi_1(x)x^2dx=\bigints_{-\infty}^{+\infty} (\dfrac{4m^3 \omega^3}{\pi \hbar^3})^{1/4}x ~ exp(-\dfrac{m \omega}{2\hbar}x^2) (\dfrac{4m^3 \omega^3}{\pi \hbar^3})^{1/4}x ~ exp(-\dfrac{m \omega}{2\hbar}x^2)x^2 dx \\
        \\
        \\
        =\sqrt{\dfrac{4 m^3 \omega^3}{\pi h^3}}\bigints_{-\infty}^{+\infty} x^4 ~ exp\left(-\dfrac{m \omega}{\hbar}x^2\right)dx \\
        \\
      $
      Let's use $\xi \equiv \sqrt{\dfrac{m \omega}{\hbar}}x$. \\
      \\
      \\
      $
        =\sqrt{\dfrac{4 \hbar}{\pi m \omega}} \bigints_{-\infty}^{+\infty} \xi^4 ~ exp(-\xi^2) \left(\sqrt{\dfrac{h}{m \omega}}\right)d\xi=\dfrac{4}{\sqrt{\pi}}\dfrac{\hbar}{m \omega} \bigints_{0}^{\infty} \xi^4 ~ exp(-\xi^2) d\xi=\dfrac{3\hbar}{2 m \omega} \\
        \\
        \\
        \therefore ~ <x^2>=\dfrac{3\hbar}{2 m \omega} ~~~ \surd \\
        \\
        \\
        \\
        \\
        <p>=\bigints_{-\infty}^{+\infty} \psi^{*}_1(x) \left(-i \hbar \dfrac{\partial}{\partial x}\right) \psi_1(x)dx \\
        \\
        =-i \hbar \bigints_{-\infty}^{+\infty} (\dfrac{4m^3 \omega^3}{\pi \hbar^3})^{1/4}x ~ exp(-\dfrac{m \omega}{2\hbar}x^2) (\dfrac{4m^3 \omega^3}{\pi \hbar^3})^{1/4} \left[exp\left(-\dfrac{m \omega}{2 \hbar}x^2\right)+x ~ exp\left(-\dfrac{m \omega}{2\hbar}x^2\right)\left(-\dfrac{m \omega}{2\hbar}x\right)\right]dx \\
        \\
        =i \hbar \sqrt{\dfrac{4 m^3 \omega^3}{\pi \hbar^3}} \bigints_{-\infty}^{+\infty} x \left(\dfrac{m \omega}{\hbar}x^2-1\right) ~ exp\left(-\dfrac{m \omega}{\hbar}x^2\right)dx=0 \\
        \\
        \\
        \therefore ~ <p>=0 ~~~ \surd \\
        \\
        \\
        \\
        \\
        <p^2>=\bigints_{-\infty}^{+\infty} \psi^{*}_1(x) \left(-i \hbar \dfrac{\partial}{\partial x}\right)^2 \psi_1(x)dx=-\hbar^2 \bigints_{-\infty}^{+\infty} \psi^{*}_1(x) \dfrac{d^2 \psi_1}{dx^2} dx \\
        \\
        \\
        =-\hbar^2 \bigints_{-\infty}^{+\infty} (\dfrac{4m^3 \omega^3}{\pi \hbar^3})^{1/4}x ~ exp(-\dfrac{m \omega}{2\hbar}x^2) (\dfrac{4m^3 \omega^3}{\pi \hbar^3})^{1/4} ~ \dfrac{d}{dx}\left(exp\left(-\dfrac{m \omega}{2\hbar}x^2\right)+exp\left(-\dfrac{m \omega}{2\hbar}x^2\right)\left(\dfrac{-m\omega}{\hbar}x\right)x\right)dx \\
        \\
        \\
        =-\hbar^2 \sqrt{\dfrac{4m^3 \omega^3}{\pi \hbar^3}} \bigints_{-\infty}^{+\infty} x \left(-\dfrac{m \omega}{\hbar}x\right)\left[3-\dfrac{m \omega}{\hbar}x^2\right] ~ exp\left(-\dfrac{m \omega}{\hbar}x^2\right) dx \\
        \\
        \\
        =\hbar^2 \sqrt{\dfrac{4m^3 \omega^3}{\pi \hbar^3}} \bigints_{-\infty}^{+\infty} x^2 \dfrac{m \omega}{\hbar} \left[3-\dfrac{m \omega}{\hbar}x^2\right] ~ exp\left(-\dfrac{m \omega}{\hbar}x^2\right) dx \\
        \\
        \\
        =\hbar^2 \sqrt{\dfrac{4m^3 \omega^3}{\pi \hbar^3}} \bigints_{-\infty}^{+\infty} \xi^2 (3-\xi^2) ~ exp(-\xi^2) \left(\sqrt{\dfrac{\hbar}{m \omega}}\right) d\xi \\
        \\
        \\
        =\dfrac{4 \hbar m \omega}{\sqrt{\pi}} \bigints_{0}^{+\infty} \xi^2 (3-\xi^2) ~ exp(-\xi^2) d\xi=\dfrac{4 \hbar m \omega}{\sqrt{\pi}} \left(\dfrac{3 \sqrt{\pi}}{8}\right) \\
        \\
        \\
        \therefore ~ <p^2>=\dfrac{3 \hbar m \omega}{2} ~~~ \surd \\
        \\
        \\
        \\
        \\
        \begin{cases}
          \sigma_x=\sqrt{<x^2>-<x>^2}=\sqrt{\dfrac{3\hbar}{2m \omega}} \\
          \\
          \sigma_p=\sqrt{<p^2>-<p>^2}=\sqrt{\dfrac{3\hbar m \omega}{2}}
        \end{cases} \Longrightarrow \sigma_x \sigma_p=\dfrac{3\hbar}{2} ~~~ \surd \\ \\
        \\
      $
      We just showed that the uncertainty principle $\geq 2$ holds. \\
      \\
      $
      \begin{cases}
        <T>=<\dfrac{p^2}{2m}>=\dfrac{1}{2m}<p^2>=\dfrac{1}{2m}\left(\dfrac{3\hbar m \omega}{2}\right)=\dfrac{3\hbar \omega}{4} \\
        \\
        <V>=<\dfrac{1}{2}m \omega^2 x^2>=\dfrac{1}{2}m \omega^2<x^2>=\dfrac{1}{2}m \omega^2 \left(\dfrac{3\hbar}{2m \omega}\right)=\dfrac{3\hbar \omega}{4}
      \end{cases} \\ \\
      $
      Now let's fnd the sum of $<T>$ and $<V>$: \\
      \\
      $
        <T>+<V>=\dfrac{3\hbar \omega}{2} \\ \\
      $
      This is the energy of the first excited state. \\
    }

  \end{itemize}

  \rule{15cm}{1pt}

  \textbf{2.12}
  \begin{itemize}
    \item Find $<x>, <p>, <x^2>, <p^2>$ and $<T>$, for the nth stationary state of the harmonic oscillator,
    using the method of Example 2.5. Check that the uncertainty principle is satisfied.
    
    \textcolor{hwColor}{
      Starting off with the operators for the harmonic oscillator that are defined as: \\
      \\
      $
        \begin{cases}
          \hat{a}_+=\dfrac{1}{\sqrt{2 m \hbar \omega}}\left(-i \hat{p}+m \omega \hat{x}\right) \Rightarrow \hat{x}=\sqrt{\dfrac{\hbar}{2 m \omega}}(\hat{a}_+ + \hat{a}_-) \\
          \\
          \hat{a}_-=\dfrac{1}{\sqrt{2 m \hbar \omega}}\left(i \hat{p}+m \omega \hat{x}\right) \Rightarrow \hat{x}=i\sqrt{\dfrac{\hbar}{2 m \omega}}(\hat{a}_+ - \hat{a}_-)
        \end{cases} \\
        \\
      $
      We know that the energy of the nth eigenstate is $E_n=\left(n+\dfrac{1}{2}\right) \hbar \omega$ \\
      \\
      $
      \begin{cases}
        \hbar \omega \left(\hat{a}_+\hat{a}_-+\dfrac{1}{2}\right) \psi=E \psi \\
        \\
        \hbar \omega \left(\hat{a}_-\hat{a}_+-\dfrac{1}{2}\right) \psi=E \psi
      \end{cases} \Rightarrow \begin{cases}
        \hbar \omega \left(\hat{a}_+\hat{a}_-+\dfrac{1}{2}\right) \psi_n=\left(n+\dfrac{1}{2}\right) \hbar \omega \psi_n \\
        \\
        \hbar \omega \left(\hat{a}_-\hat{a}_+-\dfrac{1}{2}\right) \psi_n=\left(n+\dfrac{1}{2}\right) \hbar \omega \psi_n 
      \end{cases} \\
      \\
      \\
      \Longrightarrow \begin{cases}
        \hat{a}_+\hat{a}_-\psi_n=n\psi_n \\
        \\
        \hat{a}_-\hat{a}_+\psi_n=\left(n+1\right)\psi_n
      \end{cases} ~~~~ \surd \\ \\
      $
      Now it is time to find $<x>, <p>, <x^2>, <p^2>, <T>$ and $<V>$. \\
      \\
      $
        <x>=\bigints_{-\infty}^{+\infty} \psi^{*}_n(x) \psi_n(x)\hat{x}dx=\bigints_{-\infty}^{+\infty} \psi^{*}_n(x) \psi_n(x) \left(\sqrt{\dfrac{\hbar}{2m\omega}}(\hat{a}_+ +\hat{a}_-)\right)dx \\
        \\
        \\
        =\sqrt{\dfrac{\hbar}{2m\omega}} \bigints_{-\infty}^{+\infty} \psi^{*}_n(x) \left[\hat{a}_+ \psi_n(x)+\hat{a}_- \psi_n(x)\right]dx \\
        \\
        \\
        =\sqrt{\dfrac{\hbar}{2m\omega}} \bigints_{-\infty}^{+\infty} \psi_*(x) \left[A_{n+1} \psi_{n+1}(x)+A_{n-1} \psi_{n-1}(x)\right])dx=0 \\
        \\
        \\
        \therefore ~ <x>=0 ~~~ \surd \\
        \\
        \\
        \\
        \\
        <x^2>=\bigints_{-\infty}^{+\infty} \psi^{*}_n(x) \psi_n(x)\hat{x}^2 dx=\bigints_{-\infty}^{+\infty} \psi^{*}_n(x) \left[\sqrt{\dfrac{\hbar}{2m\omega}}(\hat{a}_+ + \hat{a}_-)\right]^2 \psi_n(x) dx \\
        \\
        \\
        =\dfrac{\hbar}{2m\omega} \bigints_{-\infty}^{+\infty} \psi^*_n(x) \left[\hat{a}^2_+ \psi_n(x)+\hat{a}_+ \hat{a}_- \psi_n(x)+ \hat{a}_- \hat{a}_+ \psi_n(x)+\hat{a}^2_- \psi_n(x)\right]dx \\
        \\
        \\
        \dfrac{\hbar}{2m\omega} \bigints_{-\infty}^{+\infty} \psi^*_n(x) \left[A_{n+2} \psi_{n+2}(x)+n\psi_n(x)+(n+1)\psi_n(x)+A_{n-2}\psi_{n-2}(x)\right]dx \\
        \\
        \\
        \therefore ~ <x^2>=\dfrac{\hbar}{2m\omega} \left[2n+1\right] ~~~ \surd \\
        \\
        \\
        \\
        \\
        <p>=\bigints_{-\infty}^{+\infty} \psi^{*}_n(x) \psi_n(x)\hat{p} dx=\bigints_{-\infty}^{+\infty} \psi^{*}_n(x) \left[i\sqrt{\dfrac{\hbar m \omega}{2}}(\hat{a}_+ - \hat{a}_-)\right] \psi_n(x) dx \\
        \\
        \\
        =-i\sqrt{\dfrac{\hbar m \omega}{2}} \bigints_{-\infty}^{+\infty} \psi^{*}_n(x)\left[\hat{a}_+ \psi_n(x)- \hat{a}_- \psi_n(x)\right]dx \\
        \\
        \\
        =-i\sqrt{\dfrac{\hbar m \omega}{2}} \bigints_{-\infty}^{+\infty} \psi^{*}_n(x) \left[A_{n+1} \psi_{n+1}(x)-A_{n-1} \psi_{n-1}(x)\right] dx \\
        \\
        \\
        =i\sqrt{\dfrac{\hbar m \omega}{2}} \left[A_{n+1} \bigints_{-\infty}^{+\infty} \psi^*_n(x) \psi_{n+1}(x)-A_{n-1} \bigints_{-\infty}^{+\infty} \psi^*_n \psi_{n-1}(x)\right]dx \\
        \\
        \\
        \therefore ~ <p>=0 ~~~ \surd \\
        \\
        \\
        \\
        \\
        <p^2>=\bigints_{-\infty}^{+\infty} \psi^{*}_n(x) \psi_n(x)\hat{p}^2 dx=\bigints_{-\infty}^{+\infty} \psi^{*}_n(x) \left[i\sqrt{\dfrac{\hbar m \omega}{2}}(\hat{a}_+ - \hat{a}_-)\right]^2 \psi_n(x) dx \\
        \\
        \\
        =-\dfrac{\hbar m \omega}{2} \bigints_{-\infty}^{+\infty} \psi^{*}_n(x) \left[\hat{a}^2_+ - \hat{a}_+ \hat{a}_- -\hat{a}_- \hat{a}_+ +\hat{a}^2_-\right] \psi_n(x) dx \\
        \\
        \\
        =-\dfrac{\hbar m \omega}{2} \bigints_{-\infty}^{+\infty} \psi^{*}_n(x) \left[A_{n+2} \psi_{n+2}(x)-n \psi_n(x)-(n+1) \psi_n(x)+A_{n-2} \psi_{n-2}(x)\right] dx=-\dfrac{\hbar m \omega}{2} \left[-n-(n+1)\right] \\
        \\
        \\
        \therefore ~ <p^2>=\dfrac{\hbar m \omega}{2}(2n+1) ~~~ \surd \\
        \\
        \\
        \\
        \\
        \begin{cases}
          \sigma_x=\sqrt{<x^2>-<x>^2}=\sqrt{\dfrac{\hbar}{2m \omega}(2n+1)} \\
          \\
          \sigma_p=\sqrt{<p^2>-<p>^2}=\sqrt{\dfrac{\hbar m \omega}{2}(2n+1)}
        \end{cases} \Longrightarrow \sigma_x \sigma_p=\dfrac{\hbar}{2}(2n+1) ~~~ \surd \\ \\
        \\
        \\
        $
        Note that the uncertainty principle is satisfied for the nth eigenstate. ($\sigma_x \sigma_p \geq \dfrac{\hbar}{2}$) \\ \\
        $
        \begin{cases}
          <T>=<\dfrac{p^2}{2m}>=\dfrac{1}{2m}<p^2>=\dfrac{1}{2m}\left(\dfrac{\hbar m \omega}{2}(2n+1)\right)=\dfrac{\hbar \omega}{4}(2n+1) \\
          \\
          <V>=<\dfrac{1}{2}m \omega^2 x^2>=\dfrac{1}{2}m \omega^2<x^2>=\dfrac{1}{2}m \omega^2 \left(\dfrac{\hbar}{2m \omega}(2n+1)\right)=\dfrac{\hbar \omega}{4}(2n+1)
        \end{cases} \\ \\
        \\
        <T>+<V>=E_n
      $
    }

  \end{itemize}

  \pagebreak

  \textbf{2.13}
  \begin{itemize}
    \item A particle in the harmonic oscillator potential starts out in the state 
    $$\Psi(x, 0)=A \left[3\psi_0(x)+4\psi_1(x)\right]$$ 
    \begin{enumerate}
      \item Find $A$.

      \item Construct $\Psi(x, t) and |\Psi(x, 0)|^2$. Don't get too excited if $|\Psi(x, )|^2$ oscillates 
      at exactly the classical frequency; what would it have been had I specified $\psi_2(x)$, instead of
      $\psi_1(x)$?

      \item Find $<x>$ and $<p>$. Check that Ehrenfest's theorem (Equation 1.38) holds, for this wave function.

      \item If you measured the energy of this particle, what values might you get, and with what probabilities?

        \textcolor{hwColor}{
          Behnam was here
        }

    \end{enumerate}
  \end{itemize}

  \rule{15cm}{1pt}

  \textbf{2.14}
  \begin{itemize}
    \item In the ground state of the harmonic oscillator, what is the probability (correct to three
    significant digits) of finding the particle outside the classically allowed region?
    \emph{Hint:} Classically, the energy of an oscillator is $E=\dfrac{1}{2}ka^2=\dfrac{1}{2}m \omega a^2$,
    where $a$ is the amplitude. So the "classically allowed region" for an oscillator of energy $E$
    extends from $-\sqrt{\dfrac{2E}{m \omega^2}}$ to $+\sqrt{\dfrac{2E}{m \omega^2}}$. Look in math table
    under "Normal Distribution" or "Error Function" for the numerical value of the integral, or evaluate
    it by computer.

      \textcolor{hwColor}{
        Born's interpretation, says that the probability distribution for the particle’s position in the
        ground state at time t is given by $|\Psi_0(x,t)|^2$. Classically, the probability that the particle 
        lies in the allowed region is \\ \\ 
        $$\bigints_{-a}^{+a} |\Psi_0(x,t)|^2 dx$$ \\
        \\
        The probability that it’s not in this region is: \\
        \\
        $
          1-\bigints_{-a}^{+a} |\Psi_0(x,t)|^2 dx=1-\bigints_{-a}^{+a} \Psi_0(x,t) \Psi^*_0(x,t) dx=1-\bigints_{-a}^{+a} \left[\Psi_0(x,t) e^{-i E_0 t/\hbar}\right] \left[\Psi^*_0(x,t) e^{i E_0 t/\hbar}\right] dx \\
          \\
          \\
          =1-\bigints_{-a}^{+a} |\Psi_0(x,t)|^2 dx=1-\bigints_{-a}^{+a} \left[\left(\dfrac{m \omega}{\pi \hbar}\right)^{1/4} ~ exp\left(-\dfrac{m \omega}{2\hbar}x^2\right)\right]^2 dx \\
          \\
          \\
          =1-\sqrt{\dfrac{m \omega}{\pi \hbar}} \bigints_{-a}^{+a} ~ exp\left(-\dfrac{m \omega}{\hbar} x^2\right)dx  \\ \\ 
        $
        Like the previous problems using $\xi \equiv \sqrt{\dfrac{m \omega}{\hbar}x}$ defined variable helps us to solve this problem faster. \\
        \\
        \\
        $
          1-\bigints_{-a}^{+a} |\Psi_0(x,t)|^2 dx=1-\sqrt{\dfrac{m \omega}{\pi \hbar}} \bigint_{-\sqrt{\dfrac{m \omega}{\hbar}a}}^{+\sqrt{\dfrac{m \omega}{\hbar}a}} ~ \sqrt{\dfrac{\hbar}{m \omega}} ~ exp(-\xi^2) d\xi \\
          \\
          \\
          =1-\dfrac{2}{\sqrt{\pi}} \bigint_{0}^{+\sqrt{\dfrac{m \omega}{\hbar}a}} exp(-\xi^2) d\xi \\
          \\
        $
        The error function is $erf ~ z=\dfrac{2}{\sqrt{\pi}} \bigint_{0}^{z} exp(-x^2) dx$. So now we can rewrite 
        the probability as the following: \\
        \\
        $
          1-\bigints_{-a}^{+a} |\Psi_0(x,t)|^2 dx=1-erf(\sqrt{\dfrac{m \omega}{\hbar}a}) \\ \\
        $
        Now we cand find $a$ easily: $\dfrac{\hbar \omega}{2}=\dfrac{1}{2} m \omega^2 a^2 \Longrightarrow a=\sqrt{\dfrac{\hbar}{m \omega}}$.
        Therefore, the probability that the particle lies outside the allowed region in the ground state is: \\
        \\
        $
          1-erf(1) \approx 0.16
        $
      }

  \end{itemize}

  \rule{15cm}{1pt}

  \textbf{2.44}
  \begin{itemize}
    \item If two (or more) distinct solutions to the (time-independent) Schr$\ddot{o}$dinger equation have the
    same energy $E$, these states are said to be \textbf{degenerate}. For example the free particle states are 
    doubly degenerate -- one solution representing motion to the right, and the other motion to the left. But
    we have never encountered \emph{normalizable} degenerate solutions, and this is no accident. Prove the following,
    theorem: \emph{In one dimension ($-\infty < x < +\infty$) there are no degenerate bound states. [Hint:} suppose
    there are \emph{two} solutions, $\psi_1$ and $\psi_2$, with the same energy $E$. Multiply the Schr$\ddot{o}$dinger equation
    for $\psi_1$ by $\psi_2$, and the Schr$\ddot{o}$dinger equation for $\psi_2$ by $\psi_1$, and subtract,
    to show that $\left(\psi_2 \dfrac{d\psi_1}{dx}-\psi_1 \dfrac{d\psi_2}{dx}\right)$ is a constant. Use the 
    fact that for normalizable solutions $\psi \longrightarrow 0$ at $\pm \infty$ to demonstrate that this constant
    is in fact zero. Conclude that $\psi_2$ is a multiple of $\psi_1$, and hence that the two solutions
    are not distinct.]

      \textcolor{hwColor}{
        Saman 
      }

  \end{itemize}

\end{document}
